\documentclass[a4paper, 14pt]{extarticle}

\usepackage{../latexDependencies/misc/preamble2}

\geometry{a4paper}

% Название дисциплины
\newcommand{\subject}{Теория автоматов и алгоритмические языки} 

% Тип работы
% lab - для лабораторной работы 
% hw  - для домашней     работы
\newcommand{\task}{hw} 

% Номер работы
\newcommand{\taskNumber}{1-2} 

% Название работы
\newcommand{\taskName}{Итерационный метод Якоби для полного решения задачи} 

% Имя студента
\newcommand{\studentName}{Очкин Н.В.}

% Имя преподававателя
\newcommand{\teacherName}{Кутыркин В.А.}

% Группа
\newcommand{\group}{ФН11-52Б}

% Вариант
\newcommand{\variant}{9}

\begin{document}

\graphicspath{ {../latexDependencies/images} } 
\input{../latexDependencies/frontmatter/titlepage2}

\newgeometry{left=25mm, right=25mm, top=20mm, bottom=20mm}

% \graphicspath{ {../latexDependencies/images/LW1} }

% Customize section, subsection and subsubsection styles
\titleformat{\section}
  {\normalfont\Large\bfseries}{\thesection}{1em}{}

\titleformat{\subsection}
  {\normalfont\large\bfseries}{\thesubsection}{1em}{}

\titleformat{\subsubsection}
  {\normalfont\normalsize\bfseries}{\thesubsubsection}{1em}{}

\titleformat{\paragraph}
  {\small\normalsize\bfseries}{\theparagraph}{1em}{}

\thispagestyle{empty}

% \null\newpage

\setcounter{tocdepth}{5}
\setcounter{secnumdepth}{5}

\pagenumbering{roman}

\tableofcontents
\newpage

\pagenumbering{arabic}
\setcounter{page}{1}

\setstretch{1}
\linespread{1.1}

% \setlength{\parskip}{29pt}

\setlength{\parindent}{0pt}

\fontsize{12pt}{16pt}\selectfont

\newcommand{\OV}[1]{%
  \overset{\rule{5pt}{0.4pt}}{\text{#1}\vphantom{\text{\normalsize I}}}%
  \hspace{2pt}
}

\newcommand{\NOV}[1]{
  \text{#1}%
  \hspace{2pt}
}

% --------------------------------------START--------------------------------------

\section{Задача \rom{1}}

Найти СДНФ, СКНФ и полином Жегалкина булевой функции f от четырёх переменных,
заданной таблицей:

\begin{table}[h!]
  \centering
  \setlength{\arrayrulewidth}{0.5mm}
  \setlength{\tabcolsep}{38pt}
  \renewcommand{\arraystretch}{1.5}
  \begin{adjustbox}{max width=1\textwidth}
    \begin{tabular}{|c|c|c|c|c|c|}
    \hline
    \rowcolor{gray!50}
    \# & \bfseries A & \bfseries B & \bfseries C & \bfseries D & \bfseries \itshape f \\
    \hline
    0 & 0 & 0 & 0 & 0 & 1 \\
    \hline
    1 & 0 & 0 & 0 & 1 & 0 \\
    \hline
    2 & 0 & 0 & 1 & 0 & 0 \\
    \hline
    3 & 0 & 0 & 1 & 1 & 0 \\
    \hline
    4 & 0 & 1 & 0 & 0 & 0 \\
    \hline
    5 & 0 & 1 & 0 & 1 & 0 \\
    \hline
    6 & 0 & 1 & 1 & 0 & 1 \\
    \hline
    7 & 0 & 1 & 1 & 1 & 1 \\
    \hline
    8 & 1 & 0 & 0 & 0 & 0 \\
    \hline
    9 & 1 & 0 & 0 & 1 & 1 \\
    \hline
    10 & 1 & 0 & 1 & 0 & 0 \\
    \hline
    11 & 1 & 0 & 1 & 1 & 1 \\
    \hline
    12 & 1 & 1 & 0 & 0 & 1 \\
    \hline
    13 & 1 & 1 & 0 & 1 & 1 \\
    \hline
    14 & 1 & 1 & 1 & 0 & 0 \\
    \hline
    15 & 1 & 1 & 1 & 1 & 1 \\
    \hline
    \end{tabular}
  \end{adjustbox}
\end{table}

Если возможно, сократить запись совершенных нормальных
форм и сделать проверку, используя таблицу истинности, в том числе и для полинома
Жегалкина.

\newpage

\subsection{СДНФ}

\subsubsection{Часть 1. Метод Квайна — Мак-Класки}

\paragraph{Шаг 1: находим основные импликанты}

\vspace{10pt}

Можно легко записать СДНФ, просто просуммировав минтермы, где функция принимает значение 1. 

\begin{equation*}
  f = \overset{0}{\OV{A}\OV{B}\OV{C}\OV{D}} + \overset{6}{\OV{A}\NOV{B}\NOV{C}\OV{D}} + 
  \overset{7}{\OV{A}\NOV{B}\NOV{C}\NOV{D}} + \overset{9}{\NOV{A}\OV{B}\OV{C}\NOV{D}} + 
  \overset{11}{\NOV{A}\OV{B}\NOV{C}\NOV{D}} + \overset{12}{\NOV{A}\NOV{B}\OV{C}\OV{D}} + 
  \overset{13}{\NOV{A}\NOV{B}\OV{C}\NOV{D}} + \overset{15}{\vphantom{\OV{A}}\NOV{A}\NOV{B}\NOV{C}\NOV{D}}
\end{equation*}

\begin{equation*}
  f = \left(0,\hspace{5pt} 6,\hspace{5pt} 7,\hspace{5pt} 9,\hspace{5pt} 11,
  \hspace{5pt} 12,\hspace{5pt} 13,\hspace{5pt} 15\right)
\end{equation*}

\vspace{10pt}

Для оптимизации запишем минтермы, включая те, которые соответствуют равнодушным состояниям, 
в следующую таблицу: 

\vspace{10pt}

\begin{table}[h!]
  \centering
  \setlength{\arrayrulewidth}{0.3mm}
  \setlength{\tabcolsep}{38pt}
  \renewcommand{\arraystretch}{1.5}
  \begin{adjustbox}{max width=1\textwidth}
    \begin{tabular}{ p{4cm} p{2cm} p{5cm} }

    \multirow{2}{*}{Количество ``1''} & \multirow{2}{*}{Минтерм} & Двоичное      \\
                                      &                          & представление \\
    &&\\
    \hdashline
    0 & m0 & 0000* \\ 
    \hdashline
    \multirow{3}{*}{2} & m6 & 0110 \\
                       & m9 & 1001 \\
                       & m12 & 1100 \\
    \hdashline
    \multirow{3}{*}{3} & m7 & 0111 \\
                       & m11 & 1011 \\
                       & m13 & 1101 \\
    \hdashline
    4 & m15 & 1111 \\
    \hdashline

    \end{tabular}
  \end{adjustbox}
\end{table}

\vspace{10pt}

Теперь можно начинать комбинировать между собой минтермы, 
то есть проводить операцию склеивания. Если два минтерма 
отличаются лишь символом, который стоит в одной и той же 
позиции в обоих, заменяем этот символ на «-», это означает, 
что данный символ для нас не имеет значения. Термы, не поддающиеся 
дальнейшему комбинированию, обозначаются «*». При переходе к импликантам 
второго ранга, трактуем «-» как третье значение. Например: -110 и -100 
или -11- могут быть скомбинированы, но не -110 и 011-.

\newpage

\begin{center}
  Импликанты 1-го уровня
\end{center}

\begin{table}[h!]
  \centering
  \setlength{\arrayrulewidth}{0.3mm}
  \setlength{\tabcolsep}{38pt}
  \renewcommand{\arraystretch}{1.5}
  \begin{adjustbox}{max width=1\textwidth}
    \begin{tabular}{ p{4cm} p{2cm} p{5cm} }

    \multirow{2}{*}{Количество ``1''} & \multirow{2}{*}{Минтерм} & Двоичное      \\
                                      &                          & представление \\
    &&\\
    \hdashline
    \multirow{4}{*}{2} & m(6, 7) & 011-* \\
                       & m(9, 11) & 10-1 \\
                       & m(9, 13) & 1-01 \\
                       & m(12, 13) & 110-* \\
    \hdashline
    \multirow{3}{*}{3} & m(7, 15) & -111 \\
                       & m(11, 15) & 1-11 \\
                       & m(13, 15) & 11-1 \\
    \hdashline

    \end{tabular}
  \end{adjustbox}
\end{table}

\vspace{10pt}

\begin{center}
  Импликанты 2-го уровня
\end{center}

\begin{table}[h!]
  \centering
  \setlength{\arrayrulewidth}{0.3mm}
  \setlength{\tabcolsep}{38pt}
  \renewcommand{\arraystretch}{1.5}
  \begin{adjustbox}{max width=1\textwidth}
    \begin{tabular}{ p{4cm} p{4cm} p{5cm} }

    \multirow{2}{*}{Количество ``1''} & \multirow{2}{*}{Минтерм} & Двоичное      \\
                                      &                          & представление \\
    &&\\
    \hdashline
    \multirow{2}{*}{2} & m(9, 11, 13, 15) & 1--1* \\
                       & m(9, 13, 11, 15) & 1--1 \\
    \hdashline

    \end{tabular}
  \end{adjustbox}
\end{table}

\vspace{10pt}

Таким образом, мы получили следующую \textbf{сокращенную дизъюнктивную нормальную форму}
заданной функции:

\begin{equation*}
  f = \OV{A}\OV{B}\OV{C}\OV{D} + \OV{A}\NOV{B}\NOV{C} + \NOV{A}\NOV{B}\OV{C} +
  \NOV{A}\NOV{D}
\end{equation*}

\newpage

\paragraph{Шаг 2: таблица простых импликант}

\begin{table}[h!]
  \centering
  \setlength{\arrayrulewidth}{0.1mm}
  \setlength{\tabcolsep}{10pt}
  \renewcommand{\arraystretch}{1.5}
  \begin{adjustbox}{max width=1\textwidth}
    \begin{tabular}{|p{4cm}|c|c|c|c|c|c|c|c|c|}
    \noalign{\global\arrayrulewidth=0.6mm}
    \hline
                      & 0 & 6 & 7 & 9 & 11 & 12 & 13 & 15 & \\
    \hline
    \noalign{\global\arrayrulewidth=0.1mm}
     m0               & 1 &   &   &   &    &    &    &    & $\OV{A}\OV{B}\OV{C}\OV{D}$ \\
    \hline
     m(6, 7)          &   & 1 & 1 &   &    &    &    &    & $\OV{A}\NOV{B}\NOV{C}$ \\
    \hline
     m(12, 13)        &   &   &   &   &    & 1  &  1 &    & $\NOV{A}\NOV{B}\OV{C}$ \\
    \hline
     m(9, 11, 13, 15) &   &   &   & 1 & 1  &    &  1 &  1 & $\NOV{A}\NOV{D}$ \\
    \noalign{\global\arrayrulewidth=0.6mm}
    \hline
                      & \faStarO  &  \faStarO &  \faStarO & \faStarO  &  \faStarO  &   \faStarO &  \faStarO  &   \faStarO & \\
    \hline
    \noalign{\global\arrayrulewidth=0.1mm}
    \end{tabular}
  \end{adjustbox}
\end{table}

\subsubsection{Часть 2. Метод Петрика}

Поскольку сокращенная форма функции очень часто не является минимальной, воспользуемся
методом Петрика для нахождения всех возможных минимальных форм на основе сокращенных.\\

\vspace{10pt}

Рассмотрим таблицу простых импликант, полученную в методе Квайна — Мак-Класки.
В колонках находится различное число единиц. Например, в колонке 0 записана одна единица,
это значит, что минтерм m0 останется в функции, если ипликанта $\OV{A}\OV{B}\OV{C}\OV{D}$ 
не будет удалена. Следовательно, импликанту $\OV{A}\OV{B}\OV{C}\OV{D}$ удалять нельзя. 
Точно так же нельзя удалять и импликанту $\NOV{A}\NOV{D}$, и т.д. На этом основании импликантную
матрицу можно упростить.\\

Поскольку простые испликанты $\OV{A}\OV{B}\OV{C}\OV{D}$, $\NOV{A}\NOV{D}$, и т.д. являются
обязательными для всех вариантов тупиковых форм, то их из матрицы можно удалить. Вместе
с ними можно удалить и образующие их минтермы, так как в функции они уже содержатся за счет
импликант $\OV{A}\OV{B}\OV{C}\OV{D}$, $\NOV{A}\NOV{D}$, и т.д. В таблице эти
минтермы отмечены звездочками (под колонками).\\

После всех удалений матрица простых импликант уничтожается, что означает, что найденная
скоращенная форма является минимальной.\\ 

\newpage

\subsubsection{Проверка}

\begin{table}[h!]
  \centering
  \setlength{\arrayrulewidth}{0.5mm}
  \setlength{\tabcolsep}{20pt}
  \renewcommand{\arraystretch}{1.5}
  \begin{adjustbox}{max width=1\textwidth}
    \begin{tabular}{|c|c|c|c|c|c|c|c|c|c|}
    \hline
    \rowcolor{gray!50}
    \# & \bfseries A & \bfseries B & \bfseries C & \bfseries D & 
    $\OV{A}\OV{B}\OV{C}\OV{D}$ & $\OV{A}\NOV{B}\NOV{C}$ & $\NOV{A}\NOV{B}\OV{C}$ & 
    $\NOV{A}\NOV{D}$ & \bfseries \itshape f \\
    \hline
    0 & 0 & 0 & 0 & 0 & 1 & 0 & 0 & 0 & 1 \\
    \hline
    1 & 0 & 0 & 0 & 1 & 0& 0&0&0 & 0 \\
    \hline
    2 & 0 & 0 & 1 & 0 & 0&0&0&0 & 0 \\
    \hline
    3 & 0 & 0 & 1 & 1 & 0&0&0&0 & 0 \\
    \hline
    4 & 0 & 1 & 0 & 0 & 0&0&0&0 & 0 \\
    \hline
    5 & 0 & 1 & 0 & 1 & 0&0&0&0 & 0 \\
    \hline
    6 & 0 & 1 & 1 & 0 & 0&1&0&0 & 1 \\
    \hline
    7 & 0 & 1 & 1 & 1 & 0&2&0&0 & 1 \\
    \hline
    8 & 1 & 0 & 0 & 0 & 0&0&0&0 & 0 \\
    \hline
    9 & 1 & 0 & 0 & 1 & 0&0&0&1 & 1 \\
    \hline
    10 & 1 & 0 & 1 & 0 & 0&0&0&0 & 0 \\
    \hline
    11 & 1 & 0 & 1 & 1 & 0&0&0&1 & 1 \\
    \hline
    12 & 1 & 1 & 0 & 0 & 0&0&1&0 & 1 \\
    \hline
    13 & 1 & 1 & 0 & 1 & 0&0&1&1 & 1 \\
    \hline
    14 & 1 & 1 & 1 & 0 & 0&0&0&0 & 0 \\
    \hline
    15 & 1 & 1 & 1 & 1 & 0&0&0&1 & 1 \\
    \hline
    \end{tabular}
  \end{adjustbox}
\end{table}

Провекра сошлась.

\subsubsection{Ответ}

Единственная минимальная дизъюнктивная нормальная форма исходной функции:
\begin{equation*}
  f = \OV{A}\OV{B}\OV{C}\OV{D} + \OV{A}\NOV{B}\NOV{C} + \NOV{A}\NOV{B}\OV{C} +
  \NOV{A}\NOV{D}
\end{equation*}

\newpage

\subsection{СКНФ}

При нахождении тупиковых и минимальных КНФ булевых функций необходимо действовать в той 
же последовательности, что и в предыдщем подразделе, но с учетом того, что для инверсии 
заданной функции требуется найти все тупиковые формы. В общем случае последовательность
действий, представленная в данном подразделе состоит в следующем:

\begin{enumerate}[a)]
  \item найти СДНФ заданной функции f;
  \item записать СДНФ функции $\OV{\itshape f}$;
  \item представить функцию $\OV{\itshape f}$ в виде сокращенной ДНФ;
  \item методом Петрика найти все тупиковые формы для ДНФ функции $\OV{\itshape f}$;
  \item все тупиковые форму проинвертировать по теореме де Моргана. Получим
  список тупиковых КНФ заданной функции $\OV{\itshape f}$;
  \item выбрать из тупиковых форм все минимальные по числу вхождений аргументов.
\end{enumerate}

\subsubsection{Нахождение тупиковых и минимальных КНФ}

\paragraph{Найти СДНФ заданной функции \textit{f}}

\begin{equation*}
  f = \overset{0}{\OV{A}\OV{B}\OV{C}\OV{D}} + \overset{6}{\OV{A}\NOV{B}\NOV{C}\OV{D}} + 
  \overset{7}{\OV{A}\NOV{B}\NOV{C}\NOV{D}} + \overset{9}{\NOV{A}\OV{B}\OV{C}\NOV{D}} + 
  \overset{11}{\NOV{A}\OV{B}\NOV{C}\NOV{D}} + \overset{12}{\NOV{A}\NOV{B}\OV{C}\OV{D}} + 
  \overset{13}{\NOV{A}\NOV{B}\OV{C}\NOV{D}} + \overset{15}{\vphantom{\OV{A}}\NOV{A}\NOV{B}\NOV{C}\NOV{D}}
\end{equation*}

\begin{equation*}
  f = \left(0,\hspace{5pt} 6,\hspace{5pt} 7,\hspace{5pt} 9,\hspace{5pt} 11,
  \hspace{5pt} 12,\hspace{5pt} 13,\hspace{5pt} 15\right)
\end{equation*}

\vspace{10pt}

\paragraph{Записать СДНФ функции $\OV{\itshape f}$}

\begin{equation*}
  \OV{\itshape f} = \overset{1}{\OV{A}\OV{B}\OV{C}\NOV{D}} + \overset{2}{\OV{A}\OV{B}\NOV{C}\OV{D}} + 
  \overset{3}{\OV{A}\OV{B}\NOV{C}\NOV{D}} + \overset{4}{\OV{A}\NOV{B}\OV{C}\OV{D}} + 
  \overset{5}{\OV{A}\NOV{B}\OV{C}\NOV{D}} + \overset{8}{\NOV{A}\OV{B}\OV{C}\OV{D}} + 
  \overset{10}{\NOV{A}\OV{B}\NOV{C}\OV{D}} + \overset{14}{\NOV{A}\NOV{B}\NOV{C}\OV{D}}
\end{equation*}

\begin{equation*}
  \OV{\itshape f} = \left(1,\hspace{5pt} 2,\hspace{5pt} 3,\hspace{5pt} 4,\hspace{5pt} 5,
  \hspace{5pt} 8,\hspace{5pt} 10,\hspace{5pt} 14\right)
\end{equation*}

\vspace{10pt}

\newpage

\paragraph{Представить функцию $\OV{\itshape f}$ в виде сокращенной ДНФ}

\begin{table}[h!]
  \centering
  \setlength{\arrayrulewidth}{0.3mm}
  \setlength{\tabcolsep}{38pt}
  \renewcommand{\arraystretch}{1.5}
  \begin{adjustbox}{max width=1\textwidth}
    \begin{tabular}{ p{4cm} p{2cm} p{5cm} }

    \multirow{2}{*}{Количество ``1''} & \multirow{2}{*}{Минтерм} & Двоичное      \\
                                      &                          & представление \\
    &&\\
    \hdashline
    \multirow{4}{*}{1} & m1 & 0001 \\
                       & m2 & 0010 \\
                       & m4 & 0100 \\
                       & m8 & 1000 \\
    \hdashline
    \multirow{3}{*}{2} & m3 & 0011 \\
                       & m5 & 0101 \\
                       & m10 & 1010 \\
    \hdashline
    3 & m14 & 1110 \\
    \hdashline

    \end{tabular}
  \end{adjustbox}
\end{table}

\begin{center}
  Импликанты 1-го уровня
\end{center}

\begin{table}[h!]
  \centering
  \setlength{\arrayrulewidth}{0.3mm}
  \setlength{\tabcolsep}{38pt}
  \renewcommand{\arraystretch}{1.5}
  \begin{adjustbox}{max width=1\textwidth}
    \begin{tabular}{ p{4cm} p{2cm} p{5cm} }

    \multirow{2}{*}{Количество ``1''} & \multirow{2}{*}{Минтерм} & Двоичное      \\
                                      &                          & представление \\
    &&\\
    \hdashline
    \multirow{6}{*}{1} & m(1, 3) & 00-1* \\
                       & m(1, 5) & 0-01* \\
                       & m(2, 3) & 001-* \\
                       & m(2, 10) & -010* \\
                       & m(4, 5) & 010-* \\
                       & m(8, 10) & 10-0* \\
    \hdashline
    2 & m(10, 14) & 1-10* \\
    \hdashline

    \end{tabular}
  \end{adjustbox}
\end{table}

\vspace{10pt}

Таким образом, мы получили следующую \textbf{сокращенную дизъюнктивную нормальную форму}
инверсии заданной функции:

\begin{equation*}
  \OV{\itshape f} = \OV{A}\OV{B}\NOV{D} + \OV{A}\OV{C}\NOV{D} + \OV{A}\OV{B}\NOV{C} + \OV{B}\NOV{C}\OV{D} +
  \OV{A}\NOV{B}\OV{C} + \NOV{A}\OV{B}\OV{D} + \NOV{A}\NOV{C}\OV{D}
\end{equation*}

\newpage

\begin{table}[h!]
  \centering
  \setlength{\arrayrulewidth}{0.1mm}
  \setlength{\tabcolsep}{10pt}
  \renewcommand{\arraystretch}{1.5}
  \begin{adjustbox}{max width=1\textwidth}
    \begin{tabular}{|p{2cm}|c|c|c|c|c|c|c|c|c|}
    \noalign{\global\arrayrulewidth=0.6mm}
    \hline
                      & 1 & 2 & 3 & 4 & 5  & 8  & 10 & 14 & \\
    \hline
    \noalign{\global\arrayrulewidth=0.1mm}
     m(1, 3)          & 1 &   & 1 &   &    &    &    &    & $\OV{A}\OV{B}\NOV{D}$ \\
    \hline
     m(1, 5)          & 1 &   &   &   &  1 &    &    &    & $\OV{A}\OV{C}\NOV{D}$ \\
    \hline
     m(2, 3)          &   &  1 &  1 &   &    &   &   &    & $\OV{A}\OV{B}\NOV{C}$ \\
     \hline
     m(2, 10)         &   & 1  &   &   &    &  &  1 &    & $\OV{B}\NOV{C}\OV{D}$ \\
     \hline
     m(4, 5)          &   &   &   &  1 &  1  &   &   &    & $\OV{A}\NOV{B}\OV{C}$ \\
     \hline
     m(8, 10)         &   &   &   &   &    & 1  &  1 &    & $\NOV{A}\OV{B}\OV{D}$ \\
     \hline
     m(10, 14)        &   &   &   &   &    &   &  1 & 1   & $\NOV{A}\NOV{C}\OV{D}$ \\

    \noalign{\global\arrayrulewidth=0.6mm}
    \hline
                      &   &   &   & \faStarO  &  \faStarO  &   \faStarO &  \faStarO  &   \faStarO & \\
    \hline
    \noalign{\global\arrayrulewidth=0.1mm}
    \end{tabular}
  \end{adjustbox}
\end{table}

\paragraph{Методом Петрика найти все тупиковые формы для ДНФ функции $\OV{\itshape f}$}

После всех удалений получим упрощенную матрицу:

\begin{table}[h!]
  \centering
  \setlength{\arrayrulewidth}{0.1mm}
  \setlength{\tabcolsep}{10pt}
  \renewcommand{\arraystretch}{1.5}
  \begin{adjustbox}{max width=1\textwidth}
    \begin{tabular}{|p{2cm}|c|c|c|c|c|}
    \noalign{\global\arrayrulewidth=0.6mm}
    \hline
                      & 1 & 2 & 3 & & \\
    \hline
    \noalign{\global\arrayrulewidth=0.1mm}
     m(1, 3)          & 1 &   & 1& $\OV{A}\OV{B}\NOV{D}$ & $\varphi_1$ \\
    \hline
     m(1, 5)          & 1 &   &  & $\OV{A}\OV{C}\NOV{D}$ & $\varphi_2$ \\
    \hline
     m(2, 3)          &   &  1 &1& $\OV{A}\OV{B}\NOV{C}$ & $\varphi_3$ \\
     \hline
     m(2, 10)         &   & 1  & & $\OV{B}\NOV{C}\OV{D}$  & $\varphi_4$\\
     \hline
     m(4, 5)          &   &   &  & $\OV{A}\NOV{B}\OV{C}$ & $\varphi_5$ \\
     \hline
     m(8, 10)         &   &   &  & $\NOV{A}\OV{B}\OV{D}$ & $\varphi_6$ \\
     \hline
     m(10, 14)        &   &   &  & $\NOV{A}\NOV{C}\OV{D}$ & $\varphi_7$ \\
    \hline
    \end{tabular}
  \end{adjustbox}
\end{table}

Введем логические переменные $\varphi_1, \varphi_2, ..., \varphi_6$ (они записаны в 
дополнительной колонке в правой части таблицы). Будем считать, что $\varphi_1 = 1$, 
если простая импликанта $\OV{A}\OV{B}\NOV{D}$ входит в функцию, и $\varphi_1 = 0$,
если не входит. Аналогично и для других простых импликант. Тогда если
\begin{equation*}
  \varphi_1 + \varphi_3 = 1,
\end{equation*}
то минтерм $m_3$ входит в функцию; если $\varphi_3 + \varphi_4 = 1$, то $m_2$ входит
в функцию и т.д.\\

Условие, при котором все минтермы останутся в функции, запишется в виде
\begin{equation*}
  (\varphi_1 + \varphi_2)(\varphi_3 + \varphi_4)(\varphi_1 + \varphi_3) = 1.
\end{equation*}

Раскроем скобки и выполним все операции согласно теореме поглощения. В конечном итоге
получим
\begin{equation*}
  \overset{\Circled{1}}{\varphi_1 \varphi_3 \vphantom{\text{\normalsize I}}} + 
  \overset{\Circled{2}}{\varphi_1 \varphi_4 \vphantom{\text{\normalsize I}}} + 
  \overset{\Circled{3}}{\varphi_2 \varphi_3 \vphantom{\text{\normalsize I}}} = 1
\end{equation*}

Таким образом, мы нашли ответ на поставленную задачу, правда, пока этот ответ 
представлен в зашифрованном виде. Расшифруем его. Каждая конъюнкция в полученном 
уравнении может быть равной единице. Если $\varphi_1 \varphi_3 = 1$, то это значит, 
что в функцию должны войти простые импликанты $\OV{A}\OV{B}\NOV{D}$ и $\OV{A}\OV{B}\NOV{C}$. 
Следовательно, получили первый вариант тупиковой формы:
\begin{equation*}
  \Circled{1}: \quad \OV{\itshape f}_1 = \OV{A}\NOV{B}\OV{C} + \NOV{A}\OV{B}\OV{D} + \NOV{A}\NOV{C}\OV{D} + 
  \OV{A}\OV{B}\NOV{D} + \OV{A}\OV{B}\NOV{C},
\end{equation*}
содержащий 12 вхождений аргументов.\\ 

Аналогично находим еще две тупиковые формы:
\begin{align*}
  \Circled{2}: \quad \OV{\itshape f}_2 = \OV{A}\NOV{B}\OV{C} + \NOV{A}\OV{B}\OV{D} + \NOV{A}\NOV{C}\OV{D} + 
  \OV{A}\OV{B}\NOV{D} + \OV{B}\NOV{C}\OV{D}; \\
  \Circled{3}: \quad \OV{\itshape f}_3 = \OV{A}\NOV{B}\OV{C} + \NOV{A}\OV{B}\OV{D} + \NOV{A}\NOV{C}\OV{D} + 
  \OV{A}\OV{C}\NOV{D} + \OV{A}\OV{B}\NOV{C}.
\end{align*}

Таким образом, инверсия заданной функции имеет три тупиковые дизъюнктивные нормальные 
формы.

\paragraph{Все тупиковые форму проинвертировать по теореме де Моргана.}

\begin{align*}
  f = \left(\NOV{A} + \OV{B} + \NOV{C}\right) \left(\OV{A} + \NOV{B} + \NOV{D}\right)
  \left(\OV{A} + \OV{C} + \NOV{D}\right) \left(\NOV{A} + \NOV{B} + \OV{D}\right)
  \left(\NOV{A} + \NOV{B} + \OV{C}\right) \\[1em]
  f = \left(\NOV{A} + \OV{B} + \NOV{C}\right) \left(\OV{A} + \NOV{B} + \NOV{D}\right)
  \left(\OV{A} + \OV{C} + \NOV{D}\right) \left(\NOV{A} + \NOV{B} + \OV{D}\right)
  \left(\NOV{B} + \OV{C} + \NOV{D}\right) \\[1em]
  f = \left(\NOV{A} + \OV{B} + \NOV{C}\right) \left(\OV{A} + \NOV{B} + \NOV{D}\right)
  \left(\OV{A} + \OV{C} + \NOV{D}\right) \left(\NOV{A} + \NOV{C} + \OV{D}\right)
  \left(\NOV{A} + \NOV{B} + \OV{C}\right) \\
\end{align*}

\paragraph{Выбрать из тупиковых форм все минимальные по числу вхождений аргументов.}

Во всех трех тупиковых формах оказалось по 15 вхождений переменных, а значит они 
все являются и минимальными.

\subsubsection{Проверка}

Проверим первую из трех минимальных форм.

\begin{table}[h!]
  \centering
  \setlength{\arrayrulewidth}{0.5mm}
  \setlength{\tabcolsep}{20pt}
  \renewcommand{\arraystretch}{1.5}
  \begin{adjustbox}{max width=1\textwidth}
    \begin{tabular}{|c|c|c|c|c|c|c|c|c|c|c|}
    \hline
    \rowcolor{gray!50}
    \# & \bfseries A & \bfseries B & \bfseries C & \bfseries D & 
    $\NOV{A} + \OV{B} + \NOV{C}$ & 
    $\OV{A} + \NOV{B} + \NOV{D}$ & 
    $\OV{A} + \OV{C} + \NOV{D}$ & 
    $\NOV{A} + \NOV{B} + \OV{D}$ & 
    $\NOV{A} + \NOV{B} + \OV{C}$ & 
    \bfseries \itshape f \\
    \hline
    0 & 0 & 0 & 0 & 0 &  1&1&1&0&1& 1 \\
    \hline
    1 & 0 & 0 & 0 & 1 &  1&1&1&0&1& 0 \\
    \hline
    2 & 0 & 0 & 1 & 0 &  1&1&1&1&0& 0 \\
    \hline
    3 & 0 & 0 & 1 & 1 &  1&1&1&0&0& 0 \\
    \hline
    4 & 0 & 1 & 0 & 0 &  0&1&1&1&1& 0 \\
    \hline
    5 & 0 & 1 & 0 & 1 &  0&1&1&1&1& 0 \\
    \hline
    6 & 0 & 1 & 1 & 0 &  1&1&1&1&1& 1 \\
    \hline
    7 & 0 & 1 & 1 & 1 &  1&1&1&1&1& 1 \\
    \hline
    8 & 1 & 0 & 0 & 0 &  1&0&1&1&1& 0 \\
    \hline
    9 & 1 & 0 & 0 & 1 &  1&1&1&1&1& 1 \\
    \hline
    10 & 1 & 0 & 1 & 0 & 1&0&0&1&1& 0 \\
    \hline
    11 & 1 & 0 & 1 & 1 & 1&1&1&1&1& 1 \\
    \hline
    12 & 1 & 1 & 0 & 0 & 1&1&1&1&1& 1 \\
    \hline
    13 & 1 & 1 & 0 & 1 & 1&1&1&1&1& 1 \\
    \hline
    14 & 1 & 1 & 1 & 0 & 1&1&0&1&1& 0 \\
    \hline
    15 & 1 & 1 & 1 & 1 & 1&1&1&1&1& 1 \\
    \hline
    \end{tabular}
  \end{adjustbox}
\end{table}

Провекра сошлась.

\subsubsection{Ответ}

Три минимальные конъюктивные нормальные формы исходной функции:
\begin{align*}
  f = \left(\NOV{A} + \OV{B} + \NOV{C}\right) \left(\OV{A} + \NOV{B} + \NOV{D}\right)
  \left(\OV{A} + \OV{C} + \NOV{D}\right) \left(\NOV{A} + \NOV{B} + \OV{D}\right)
  \left(\NOV{A} + \NOV{B} + \OV{C}\right) \\[1em]
  f = \left(\NOV{A} + \OV{B} + \NOV{C}\right) \left(\OV{A} + \NOV{B} + \NOV{D}\right)
  \left(\OV{A} + \OV{C} + \NOV{D}\right) \left(\NOV{A} + \NOV{B} + \OV{D}\right)
  \left(\NOV{B} + \OV{C} + \NOV{D}\right) \\[1em]
  f = \left(\NOV{A} + \OV{B} + \NOV{C}\right) \left(\OV{A} + \NOV{B} + \NOV{D}\right)
  \left(\OV{A} + \OV{C} + \NOV{D}\right) \left(\NOV{A} + \NOV{C} + \OV{D}\right)
  \left(\NOV{A} + \NOV{B} + \OV{C}\right) \\
\end{align*}

\newgeometry{left=25mm, right=25mm, top=20mm, bottom=20mm}

\subsection{Полином Жегалкина}

\subsubsection{Метод треугольника}

Метод треугольника (часто называемый методом треугольника Паскаля) 
позволяет преобразовать таблицу истинности в полином Жегалкина путём 
построения вспомогательной треугольной таблицы в соответствии со следующими правилами: 

\begin{itemize}
  \item Строится полная таблица истинности, в которой строки идут в порядке возрастания двоичных кодов от $000 ... 00$ до $111 ... 11$.
  \item Строится вспомогательная треугольная таблица, в которой первый столбец совпадает со столбцом значений функции в таблице истинности.
  \item Ячейка в каждом последующем столбце получается путём суммирования по модулю 2 двух ячеек предыдущего столбца — стоящей в той же строке и строкой ниже.
  \item Столбцы вспомогательной таблицы нумеруются двоичными кодами в том же порядке, что и строки таблицы истинности.
  \item Каждому двоичному коду ставится в соответствие один из членов полинома Жегалкина в зависимости от позиций кода, в которых стоят единицы. Например, ячейке 111 соответствует член ABC, ячейке 101 — член AC, ячейке 010 — член B, ячейке 000 — член 1 и т. д.
  \item Если в верхней строке какого-либо столбца стоит единица, то соответствующий член присутствует в полиноме Жегалкина.
\end{itemize}

\begin{center}
  \begin{adjustbox}{max width=1\textwidth}
    \begin{tabular}{|c|c|c|c|c|c|}
      \hline
      \# & \bfseries A & \bfseries B & \bfseries C & \bfseries D & \bfseries \itshape f \\
      \hline
      0 & 0 & 0 & 0 & 0 & 1 \\
      \hline
      1 & 0 & 0 & 0 & 1 & 0 \\
      \hline
      2 & 0 & 0 & 1 & 0 & 0 \\
      \hline
      3 & 0 & 0 & 1 & 1 & 0 \\
      \hline
      4 & 0 & 1 & 0 & 0 & 0 \\
      \hline
      5 & 0 & 1 & 0 & 1 & 0 \\
      \hline
      6 & 0 & 1 & 1 & 0 & 1 \\
      \hline
      7 & 0 & 1 & 1 & 1 & 1 \\
      \hline
      8 & 1 & 0 & 0 & 0 & 0 \\
      \hline
      9 & 1 & 0 & 0 & 1 & 1 \\
      \hline
      10 & 1 & 0 & 1 & 0 & 0 \\
      \hline
      11 & 1 & 0 & 1 & 1 & 1 \\
      \hline
      12 & 1 & 1 & 0 & 0 & 1 \\
      \hline
      13 & 1 & 1 & 0 & 1 & 1 \\
      \hline
      14 & 1 & 1 & 1 & 0 & 0 \\
      \hline
      15 & 1 & 1 & 1 & 1 & 1 \\
      \hline
    \end{tabular}
    \hspace{10pt}\renewcommand{\arraystretch}{1.017}
    \begin{tabular}{|c|c|c|c|c|c|c|c|c|c|c|c|c|c|c|c|c|c}
      \cline{1-17}
      1 & D & C & CD & B & BD & BC & BCD & A & AD & AC & ACD & AB & ABD & ABC & ABCD \\
      \cline{1-17}
      1 & 1 & 1 & 1 & 1 & 1 & 0 & 1 & 1 & 0 & 1 & 1 & 0 & 0 & 1 & 0 \\
      \cline{1-16}
      0 & 0 & 0 & 0 & 0 & 1 & 1 & 0 & 1 & 1 & 0 & 1 & 0 & 1 & 1 \\
      \cline{1-15}
      0 & 0 & 0 & 0 & 1 & 0 & 1 & 1 & 0 & 1 & 1 & 1 & 1 & 0 \\
      \cline{1-14}
      0 & 0 & 0 & 1 & 1 & 1 & 0 & 1 & 1 & 0 & 0 & 0 & 1 \\
      \cline{1-13}
      0 & 0 & 1 & 0 & 0 & 1 & 1 & 0 & 1 & 0 & 0 & 1 \\
      \cline{1-12}
      0 & 1 & 1 & 0 & 1 & 0 & 1 & 1 & 1 & 0 & 1 \\
      \cline{1-11}
      1 & 0 & 1 & 1 & 1 & 1 & 0 & 0 & 1 & 1 \\
      \cline{1-10}
      1 & 1 & 0 & 0 & 0 & 1 & 0 & 1 & 0 \\
      \cline{1-9}
      0 & 1 & 0 & 0 & 1 & 1 & 1 & 1 \\
      \cline{1-8}
      1 & 1 & 0 & 1 & 0 & 0 & 0 \\
      \cline{1-7}
      0 & 1 & 1 & 1 & 0 & 0 \\
      \cline{1-6}
      1 & 0 & 0 & 1 & 0 \\
      \cline{1-5}
      1 & 0 & 1 & 1 \\
      \cline{1-4}
      1 & 1 & 0 \\
      \cline{1-3}
      0 & 1 \\
      \cline{1-2}
      1 \\
      \cline{1-1}
    \end{tabular}
  \end{adjustbox}
\end{center}

\begin{equation*}
  f(\text{A}, \text{B}, \text{C}, \text{D}) = 1 \oplus \NOV{D} \oplus \NOV{C} \oplus
  \NOV{C}\NOV{D} \oplus \NOV{B} \oplus \NOV{B}\NOV{D} \oplus \NOV{B}\NOV{C}\NOV{D} \oplus
  \NOV{A} \oplus \NOV{A}\NOV{C} \oplus \NOV{A}\NOV{C}\NOV{D} \oplus \NOV{A}\NOV{B}\NOV{C}
\end{equation*}

\newpage

В качестве проверки посчитаем полином еще раз другим способом.

\subsubsection{Метод БПФ}

Наиболее экономным с точки зрения объёма вычислений и целесообразным для построения 
полинома Жегалкина вручную является метод быстрого преобразования Фурье (БПФ).\\

Строим таблицу, состоящую из 2N столбцов и N + 1 строк, где N — количество переменных в 
функции. В верхней строке таблицы размещаем вектор значений функции, то есть последний 
столбец таблицы истинности.\\

Каждую строку полученной таблицы разбиваем на блоки (черные линии на рисунке). 
В первой строке блок занимает одну клетку, во второй строке — две, в третьей — четыре, 
в четвёртой — восемь и т. д. Каждому блоку в некоторой строке, который мы будем называть 
«нижний блок», всегда соответствует ровно два блока в предыдущей строке. Будем называть 
их «левый верхний блок» и «правый верхний блок».\\

Построение начинается со второй строки. Содержимое левых верхних блоков без 
изменения переносится в соответствующие клетки нижнего блока. Затем над правым верхним и 
левым верхним блоками побитно производится операция «сложение по модулю два», и полученный 
результат переносится в соответствующие клетки правой части нижнего блока. Эта операция проводится 
со всеми строками сверху вниз и со всеми блоками в каждой строке. После окончания построения 
в нижней строке оказывается строка чисел, которая является коэффициентами полинома 
Жегалкина, записанными в той же последовательности, что и в описанном выше методе треугольника. 

\begin{center}
  \renewcommand{\arraystretch}{1.5}
  \begin{adjustbox}{max width=1\textwidth}
    \begin{tabular}{|c|c|c|c|c|c|c|c|c|c|c|c|c|c|c|c|}
      \hline
      1 & 0 & 0 & 0 & 0 & 0 & 1 & 1 & 0 & 1 & 0 & 1 & 1 & 1 & 0 & 1 \\
      \hline
      \multicolumn{2}{|c|}{1 \hspace{1pt} \hfill \hspace{1pt} 1} & 
      \multicolumn{2}{ c|}{0 \hspace{1pt} \hfill \hspace{1pt} 0} & 
      \multicolumn{2}{ c|}{0 \hspace{1pt} \hfill \hspace{1pt} 0} & 
      \multicolumn{2}{ c|}{1 \hspace{1pt} \hfill \hspace{1pt} 0} & 
      \multicolumn{2}{ c|}{0 \hspace{1pt} \hfill \hspace{1pt} 1} & 
      \multicolumn{2}{ c|}{0 \hspace{1pt} \hfill \hspace{1pt} 1} & 
      \multicolumn{2}{ c|}{1 \hspace{1pt} \hfill \hspace{1pt} 0} & 
      \multicolumn{2}{ c|}{0 \hspace{1pt} \hfill \hspace{1pt} 1} \\
      \hline
      \multicolumn{4}{|c }{1 \hspace{1pt} \hfill \hspace{1pt} 1 \hspace{1pt} \hfill \hspace{1pt} 
                           1 \hspace{1pt} \hfill \hspace{1pt} 1} & 
      \multicolumn{4}{|c }{0 \hspace{1pt} \hfill \hspace{1pt} 0 \hspace{1pt} \hfill \hspace{1pt} 
                           1 \hspace{1pt} \hfill \hspace{1pt} 0} & 
      \multicolumn{4}{|c }{0 \hspace{1pt} \hfill \hspace{1pt} 1 \hspace{1pt} \hfill \hspace{1pt} 
                           0 \hspace{1pt} \hfill \hspace{1pt} 0} & 
      \multicolumn{4}{|c|}{1 \hspace{1pt} \hfill \hspace{1pt} 0 \hspace{1pt} \hfill \hspace{1pt} 
                           1 \hspace{1pt} \hfill \hspace{1pt} 1} \\
      \hline
      \multicolumn{8}{|c }{1 \hspace{1pt} \hfill \hspace{1pt} 1 \hspace{1pt} \hfill \hspace{1pt} 
                           1 \hspace{1pt} \hfill \hspace{1pt} 1 \hspace{1pt} \hfill \hspace{1pt} 
                           1 \hspace{1pt} \hfill \hspace{1pt} 1 \hspace{1pt} \hfill \hspace{1pt} 
                           0 \hspace{1pt} \hfill \hspace{1pt} 1} &
      \multicolumn{8}{|c|}{0 \hspace{1pt} \hfill \hspace{1pt} 1 \hspace{1pt} \hfill \hspace{1pt} 
                           0 \hspace{1pt} \hfill \hspace{1pt} 0 \hspace{1pt} \hfill \hspace{1pt} 
                           1 \hspace{1pt} \hfill \hspace{1pt} 1 \hspace{1pt} \hfill \hspace{1pt} 
                           1 \hspace{1pt} \hfill \hspace{1pt} 1} \\
      \hline
      \multicolumn{16}{|c|}{1 \hspace{1pt} \hfill \hspace{1pt} 1 \hspace{1pt} \hfill \hspace{1pt} 
                            1 \hspace{1pt} \hfill \hspace{1pt} 1 \hspace{1pt} \hfill \hspace{1pt} 
                            1 \hspace{1pt} \hfill \hspace{1pt} 1 \hspace{1pt} \hfill \hspace{1pt} 
                            0 \hspace{1pt} \hfill \hspace{1pt} 1 \hspace{1pt} \hfill \hspace{1pt} 
                            1 \hspace{1pt} \hfill \hspace{1pt} 0 \hspace{1pt} \hfill \hspace{1pt} 
                            1 \hspace{1pt} \hfill \hspace{1pt} 1 \hspace{1pt} \hfill \hspace{1pt} 
                            0 \hspace{1pt} \hfill \hspace{1pt} 0 \hspace{1pt} \hfill \hspace{1pt} 
                            1 \hspace{1pt} \hfill \hspace{1pt} 0} \\
      \noalign{\global\arrayrulewidth=0.8mm}
      \hline
      \hphantom{|A}1\hphantom{A|} & \hphantom{|A}D\hphantom{A|} & \hphantom{|A}C\hphantom{A|} & 
      \hphantom{A}CD\hphantom{A}  & \hphantom{|A}B\hphantom{A|} & \hphantom{A}BD\hphantom{A}  & 
      \hphantom{A}BC\hphantom{A}  & \hphantom{|}BCD\hphantom{|} & \hphantom{|A}A\hphantom{A|} & 
      \hphantom{A}AD\hphantom{A}  & \hphantom{A}AC\hphantom{A}  & \hphantom{|}ACD\hphantom{|} & 
      \hphantom{A}AB\hphantom{A}  & \hphantom{|}ABD\hphantom{|} & \hphantom{|}ABC\hphantom{|} & 
      ABCD \\
      \hline
      \noalign{\global\arrayrulewidth=0.1mm}
    \end{tabular}
  \end{adjustbox}
\end{center}

\begin{equation*}
  f(\text{A}, \text{B}, \text{C}, \text{D}) = 1 \oplus \NOV{D} \oplus \NOV{C} \oplus
  \NOV{C}\NOV{D} \oplus \NOV{B} \oplus \NOV{B}\NOV{D} \oplus \NOV{B}\NOV{C}\NOV{D} \oplus
  \NOV{A} \oplus \NOV{A}\NOV{C} \oplus \NOV{A}\NOV{C}\NOV{D} \oplus \NOV{A}\NOV{B}\NOV{C}
\end{equation*}

\subsubsection{Проверка}

Результаты в методах треугольника и БПФ совпали. 

\newpage

\section{Задача \rom{2}}

Для булевой функции из задачи \rom{1}: \\

a) используя метод каскадов, построить контактную
схему, проанализировав и проверив правильность её построения с помощью дерева
анализа;\\

б) положив $X_{(N \hspace{3pt} mod \hspace{3pt} 4)+1} = 0$, построить логическую схему из функциональных
элементов (инвертора, дизьюнктора, коньюнктора и дублятора).

\subsection{Часть а)}

\subsubsection{Синтез контактной схемы методом каскадов}

\begin{equation*}
  f(\text{A}, \text{B}, \text{C}, \text{D}) = 1 \oplus \NOV{D} \oplus \NOV{C} \oplus
  \NOV{C}\NOV{D} \oplus \NOV{B} \oplus \NOV{B}\NOV{D} \oplus \NOV{B}\NOV{C}\NOV{D} \oplus
  \NOV{A} \oplus \NOV{A}\NOV{C} \oplus \NOV{A}\NOV{C}\NOV{D} \oplus \NOV{A}\NOV{B}\NOV{C}
\end{equation*}

\begin{center}
  \scalebox{0.7}{
    \begin{tikzpicture}[
      node distance=2cm,
      neuron/.style={circle, draw, minimum size=1cm},
      arrow/.style={-{Stealth}}
  ]

      % Input layer
      \node[neuron] (q) {$q$};

      % Hidden layer 1
      \node[neuron, right=of q, yshift=3cm] (v1) {$v_1$};
      \node[above=0.2cm of v1] (v1_text) {$f(1, B, C, D)$};
      \node[neuron, right=of q, yshift=-3cm] (v2) {$v_2$};
      \node[below=0.2cm of v2] (v2_text) {$f(0, B, C, D)$};

      % Hidden layer 2
      \node[neuron, right=of v1, xshift=1cm, yshift=2cm] (v3) {$v_3$};
      \node[above=0.2cm of v3] (v3_text) {$f(1, 1, C, D)$};
      \node[neuron, right=of v1, xshift=1cm, yshift=-2cm] (v4) {$v_4$};
      \node[below=0.2cm of v4] (v4_text) {$f(1, 0, C, D)$};

      % Hidden layer 3
      \node[neuron, right=of v2, xshift=1cm, yshift=1cm] (v5) {$v_5$};
      \node[below=0.2cm of v5] (v5_text) {$f(0, 1, C, D)$};
      \node[neuron, right=of v2, xshift=1cm, yshift=-2cm] (v6) {$v_6$};
      \node[below=0.2cm of v6] (v6_text) {$f(0, 0, C, D)$};

      % Hidden layer 4
      \node[neuron, xshift=1cm, yshift=1cm, right=of v6] (v7) {$v_7$};
      \node[below=0.2cm of v7] (v7_text) {$f(0, 0, 0, D)$};

      % Output layer
      \node[neuron, right=of v4, xshift=4.5cm, yshift=-0.8cm] (e) {$e$};

      % Connections
      \draw[arrow] (q) -- (v1) node[midway, above, sloped] {$\NOV{A}$};
      \draw[arrow] (q) -- (v2) node[midway, below, sloped] {$\OV{A}$};

      \draw[arrow] (v1) -- (v3) node[midway, above, sloped] {$\NOV{B}$};
      \draw[arrow] (v1) -- (v4) node[midway, below, sloped] {$\OV{B}$};

      \draw[arrow] (v3) -- (v4) node[midway, above, sloped] {$\NOV{C}$};
      \draw[arrow] (v3) -- (e) node[midway, above, sloped] {$\OV{C}$};

      \draw[arrow] (v4) -- (e) node[midway, above, sloped] {$\NOV{D}$};

      \draw[arrow] (v2) -- (v5) node[midway, above, sloped] {$\NOV{B}$};
      \draw[arrow] (v2) -- (v6) node[midway, below, sloped] {$\OV{B}$};

      \draw[arrow] (v5) -- (e) node[midway, above, sloped] {$\NOV{C}$};

      \draw[arrow] (v6) -- (v7) node[midway, above, sloped] {$\OV{C}$};

      \draw[arrow] (v7) -- (e) node[midway, above, sloped] {$\OV{D}$};

    \end{tikzpicture}
  }
\end{center}
\begin{flalign*}
  & f(1, B, C, D) = \NOV{D} \oplus \NOV{B} \oplus \NOV{B}\NOV{D} \oplus \NOV{B}\NOV{C}\NOV{D} \oplus \NOV{B}\NOV{C}& \\
  & f(0, B, C, D) = 1 \oplus \NOV{D} \oplus \NOV{C} \oplus \NOV{C}\NOV{D} \oplus \NOV{B} \oplus \NOV{B}\NOV{D} \oplus \NOV{B}\NOV{C}\NOV{D}& \\
  & f(1, 1, C, D) = 1 \oplus \NOV{C} \oplus \NOV{C}\NOV{D}& \\
  & f(1, 0, C, D) = D& \\
  & f(0, 1, C, D) = C& \\
  & f(0, 0, C, D) = 1 \oplus \NOV{D} \oplus \NOV{C} \oplus \NOV{C}\NOV{D}& \\
  & f(0, 0, 0, D) = 1 \oplus \NOV{D}&
\end{flalign*}

\newpage

\subsubsection{Дерево анализа контактной схемы}

\vfill

\begin{center}
  \begin{tikzpicture}[
    level 1/.style={sibling distance=70mm},
    level 2/.style={sibling distance=20mm},
    level 3/.style={sibling distance=15mm},
    edge from parent/.style={draw, -latex},
    every node/.style={draw, circle},
    label node/.style={draw=none, fill=none, font=\footnotesize, text=RED}
  ]
  
    \node (q) {q}
      child {node (v1) {$v_1$}
          child {node (v3) {$v_3$}
              child {node (v4) {$v_4$}}
              child {node (v1b) {$v_1$}}
              child {node (e1) {e}}
          }
          child {node (qb) {q}}
          child {node (v4b) {$v_4$}
              child {node (v1c) {$v_1$}}
          }
          child {node (e4) {e}}
      }
      child {node (v2) {$v_2$}
          child {node (v5) {$v_5$}
              child {node (e2) {e}}
              child {node (v2b) {$v_2$}}
          }
          child {node (qc) {q}}
          child {node (v6) {$v_6$}
              child {node (v7) {$v_7$}
                  child {node (e3) {e}}
                  child {node (v6b) {$v_6$}}
              }
              child {node (v2c) {$v_2$}}
          }
      };
  
    % Adding labels separately
    \path (q) -- (v1) node[midway, above, label node] {$\NOV{A}$};
    \path (q) -- (v2) node[midway, above, label node] {$\OV{A}$};
  
    \path (v1) -- (v3) node[midway, above, label node] {$\NOV{B}$};
    \path (v1) -- (v4b) node[midway, left, label node] {$\OV{B}$};
    \path (v1) -- (qb) node[midway, left, label node] {$\NOV{A}$};
    \path (v1) -- (e4) node[midway, above, label node] {$\NOV{D}$};
  
    \path (v3) -- (v4) node[midway, left, label node] {$\NOV{C}$};
    \path (v3) -- (e1) node[midway, right, label node] {$\OV{C}$};
    \path (v3) -- (v1b) node[midway, left, label node] {$\NOV{B}$};
  
    \path (v4b) -- (v1c) node[midway, left, label node] {$\OV{B}$};
  
    \path (v2) -- (v5) node[midway, above, label node] {$\NOV{B}$};
    \path (v2) -- (v6) node[midway, above, label node] {$\OV{B}$};
    \path (v2) -- (qc) node[midway, left, label node] {$\OV{A}$};
  
    \path (v5) -- (e2) node[midway, left, label node] {$\NOV{C}$};
    \path (v5) -- (v2b) node[midway, right, label node] {$\NOV{B}$};
  
    \path (v6) -- (v7) node[midway, left, label node] {$\OV{C}$};
    \path (v6) -- (v2c) node[midway, right, label node] {$\OV{B}$};
  
    \path (v7) -- (e3) node[midway, left, label node] {$\OV{D}$};
    \path (v7) -- (v6b) node[midway, right, label node] {$\OV{C}$};
  
  \end{tikzpicture}
\end{center}

% \vfill

\begin{align*}
  & \pi_1 = \left(q   \overset{\NOV{A}}{\xrightarrow{\hspace*{1cm}}} v_1, 
                v_1 \overset{\NOV{D}}{\xrightarrow{\hspace*{1cm}}}  e \right) \\[2em]
  & \pi_2 = \left(q   \overset{\NOV{A}}{\xrightarrow{\hspace*{1cm}}} v_1, 
                v_1 \overset{\NOV{B}}{\xrightarrow{\hspace*{1cm}}}  v_3,
                v_3 \overset{\OV{C}}{\xrightarrow{\hspace*{1cm}}}  e \right) \\[2em]
  & \pi_3 = \left(q   \overset{\OV{A}}{\xrightarrow{\hspace*{1cm}}} v_2, 
                v_2 \overset{\NOV{B}}{\xrightarrow{\hspace*{1cm}}}  v_5,
                v_5 \overset{\NOV{C}}{\xrightarrow{\hspace*{1cm}}}  e \right) \\[2em]
  & \pi_4 = \left(q   \overset{\OV{A}}{\xrightarrow{\hspace*{1cm}}} v_2, 
                v_2 \overset{\OV{B}}{\xrightarrow{\hspace*{1cm}}}  v_6,
                v_6 \overset{\OV{C}}{\xrightarrow{\hspace*{1cm}}}  v_7,
                v_7 \overset{\OV{D}}{\xrightarrow{\hspace*{1cm}}}  e \right)
\end{align*}

\vfill

\begin{equation*}
  \scalebox{1.5}{$f = \OV{A}\OV{B}\OV{C}\OV{D} + \OV{A}\NOV{B}\NOV{C} + \NOV{A}\NOV{B}\OV{C} + \NOV{A}\NOV{D}$}
\end{equation*}

\vfill

\newpage

\subsection{Часть б)}

\subsubsection{Логическая схема из функциональных элементов}

\begin{gather*}
  X_{(9 \hspace{3pt} mod \hspace{3pt} 4) + 1} = X_2 = 0 \Longrightarrow \NOV{B} = 0 \\[1em]
  f = \OV{A}\OV{C}\OV{D} + \NOV{A}\NOV{D}
\end{gather*}

\begin{adjustbox}{max width=1\textwidth}
  \begin{tikzpicture}[
    node distance=2cm,
    every node/.style={draw, circle, minimum size=1cm},
    every edge/.style={draw, -latex},
    label node/.style={draw=none, fill=none, font=\footnotesize, text=RED}
  ]

    % First level nodes
    \node (A) {A};
    \node[below=of A] (C) {C};
    \node[below=of C] (D) {D};

    % Second level nodes
    \node[right=of A, yshift=3cm, xshift=1cm] (eq1) {$=$};
    \node[below=of eq1] (neg1) {$\neg$};
    \node[below=of neg1] (neg2) {$\neg$};
    \node[below=of neg2] (eq2) {$=$};
    \node[below=of eq2] (neg3) {$\neg$};

    % Third level nodes
    \node[right=of eq1, yshift=-4.5cm, xshift=2cm] (land1) {$\land$};
    \node[below=of land1] (land2) {$\land$};

    % Fourth level node
    \node[right=of land1, yshift=-1.5cm, xshift=3cm] (lor) {$\lor$};

    % Triangle node
    \node[right=of lor, xshift=0cm, draw=none, inner sep=-20pt, outer sep=-11pt] (triangle) {
      \tikz \draw[stroke=black, rotate=30, scale=1.1] (0,0) -- (0.5,0) -- (0.25,0.433) -- cycle;
    };
    
    \node[right=of triangle, xshift=-65.5pt, draw=none, inner sep=-20pt, outer sep=-2pt] (triangle2) {
      \tikz \draw[stroke=black, rotate=30, scale=1.1] (0,0) -- (0.5,0) -- (0.25,0.433) -- cycle;
    };

    % Connections
    \draw (A) -- (eq1);
    \draw (A) -- (neg1);
    \draw (C) -- (neg2);
    \draw (D) -- (eq2);
    \draw (D) -- (neg3);

    \draw (eq1) -- node[near start, above, label node] {$\NOV{A}$} (land1);
    \draw (neg1) -- node[near start, above, label node] {$\OV{A}$} (land2);
    \draw (neg2) -- node[near start, above, label node] {$\OV{C}$} (land2);
    \draw (eq2) -- node[near start, above, left, label node] {$\NOV{D}$} (land1);
    \draw (neg3) -- node[near start, above,left, label node] {$\OV{D}$} (land2);

    \draw (land1) -- (lor);
    \draw (land2) -- (lor);

    \draw (lor) -- (triangle);

  \end{tikzpicture}
\end{adjustbox}

\end{document}
