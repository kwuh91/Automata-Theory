\documentclass[letterpaper, 11pt]{extarticle}

\usepackage{../latexDependencies/misc/preamble}
\graphicspath{ {../latexDependencies/images/HW1-3} }

\newcommand{\subject}{Теория Автоматов и Алгоритмические Языки}
\newcommand{\task}{Домашняя работа \textnumero 1-1}
\newcommand{\name}{Очкин Никита}
\newcommand{\group}{ФН11-52Б}
\newcommand{\variant}{Вариант \textnumero 9}

\linespread{1}

\begin{document}

\pagestyle{empty}

% Title page
\begin{titlepage}
	\centering
	
	~
	
	\vspace{24pt}
	{\scshape\Huge \subject \par}
	\vspace{6pt}
	{\scshape\large \task \par}
	\vspace{\stretch{1.25}}
	{\itshape\large \par}
	\vspace{6pt}
	{\itshape\Large \name \par}
    \vspace{6pt}
    {\itshape\Large \variant \par}
    \vspace{\stretch{6}}
	{\large \par}
\end{titlepage}


\section*{Задача \rom{1}}

\begin{spacing}{1.5}
По номеру $ \text{\textnumero}^{\stackrel{A}{=}} (\alpha) =  1335 - 36 \cdot N \cdot (53 - n)$ 
слова $\alpha \in A^{+}$ в натуральном словаре позитивного языка над алфавитом 
\matr{A} = $\left<a,b,c,d,e,f\right>$ определить это слово и проверить результат вычисления.
\end{spacing}

\section*{Дано}

\begin{equation*}
    N = 9, \quad n = 52, \quad \text{\textnumero}^{\stackrel{A}{=}} (\alpha) = 1011
\end{equation*}

\section*{Решение}

Опишем алгоритм нахождения слова в виде псевдокода
\begin{tcolorbox}[colback=gray!10, colframe=gray!50, title=Псевдокод алгоритма нахождения слова по его номеру в лексикографическом словаре]
    \begin{verbatim}
    ввод: число, алфавит
    условие: число > 0
    вывод: результирующее слово
    
    ответ = пустая строка
    длина = размер алфавита

    пока число не равно нулю:
        номерБуквы = число мод длина
        если номерБуквы равен 0:
            вычесть длину из числа

        буква = алфавит по индексу (номерБуквы - 1) мод длина
        добавить ответ к букве

        разделить число на длину (целочисленное деление) и обновить число

    вернуть перевернутый ответ

    \end{verbatim}
\end{tcolorbox}

\noindent Вычисления произведем в python:

\newpage

\noindent алфавит: ['a', 'b', 'c', 'd', 'e', 'f'] \\ 
длина алфавита: 6

\begin{align*}
    \begin{aligned}
        & \text{Номер итерации: }& 1\\
        & \text{Текущее значение: }& 1011\\
        & 1011 \quad \% \quad 6 =&  3 \\
        & \text{Текущий индекс буквы в словаре: }& 2\\
        & \text{Текущая буква: }& \text{c}\\
        & \text{Текущее слово: }& \text{c}\\
    \end{aligned}
    \qquad
    \begin{aligned}
        & \text{Номер итерации: }& 2\\
        & \text{Текущее значение: }& 168\\
        & 168 \quad \% \quad 6 =&  0 \\
        & \text{Новое текущее значение: }& 162\\
        & \text{Текущий индекс буквы в словаре: }& 5\\
        & \text{Текущая буква: }& \text{f}\\
        & \text{Текущее слово: }& \text{fc}\\
    \end{aligned}
\end{align*}

\begin{align*}
    \begin{aligned}
        & \text{Номер итерации: }& 3\\
        & \text{Текущее значение: }& 27\\
        & 27 \quad \% \quad 6 =&  3 \\
        & \text{Текущий индекс буквы в словаре: }& 2\\
        & \text{Текущая буква: }& \text{c}\\
        & \text{Текущее слово: }& \text{cfc}\\
    \end{aligned}
    \qquad
    \begin{aligned}
        & \text{Номер итерации: }& 4\\
        & \text{Текущее значение: }& 4\\
        & 4 \quad \% \quad 6 =&  4 \\
        & \text{Текущий индекс буквы в словаре: }& 3\\
        & \text{Текущая буква: }& \text{d}\\
        & \text{Текущее слово: }& \text{dcfc}\\
    \end{aligned}
\end{align*}\\

\noindent Итого мы получили слово \guillemotleft dcfc \guillemotright из номера 1011 за 4 итерации.\\

\begin{spacing}{1.5}
\noindent Для проверки результата воспользуемся библиотекой \high{itertools}, с помощью которой
найдем все возможные слова вплоть до \guillemotleft ffffff \guillemotright 
и упорядоченно их запишем:\\
\end{spacing}

\vspace{-10pt}

\begin{gather*}
    [ \text{a}, \text{b}, \text{c}, \text{d}, \text{e}, \text{f}, \text{aa}, 
    \text{ab}, \text{ac}, \text{ad}, \text{ae}, \text{af}, \text{ba}, \text{bb}, 
    \text{bc}, \text{bd}, \text{be}, \text{bf}, \text{ca}, \text{cb}, \text{cc}, 
    \text{cd}, \text{ce}, \\
    \text{cf}, \text{da}, \text{db}, \text{dc}, \text{dd}, \text{de}, \text{df}, 
    \text{ea}, \text{eb}, \text{ec}, \text{ed}, \text{ee}, \text{ef}, \text{fa}, 
    \text{fb}, \text{fc}, \text{fd}, \text{fe}, \text{ff}, \text{aaa}, \text{aab}, \\
    \dots \\
    \text{cccccd}, \text{ccccce}, \text{cccccf}, \text{ccccda}, \text{ccccdb}, 
    \text{ccccdc}, \text{ccccdd}, \text{ccccde}, \text{ccccdf}, \\ 
    \text{ccccea}, \text{cccceb}, \text{ccccec}, \text{cccced}, \text{ccccee},  
    \text{ccccef}, \text{ccccfa}, \text{ccccfb}, \text{ccccfc},  \\ 
    \dots \\
    \text{ffffda}, \text{ffffdb}, \text{ffffdc} \text{ffffdd}, \text{ffffde}, 
    \text{ffffdf}, \text{ffffea}, \text{ffffeb}, \text{ffffec}, \\
    \text{ffffed}, \text{ffffee}, \text{ffffef}, \text{fffffa}, \text{fffffb}, 
    \text{fffffc}, \text{fffffd}, \text{fffffe}, \text{ffffff} ]
\end{gather*}\\

\noindent Теперь же выведем 1011 - 1 й элемент и получим \guillemotleft dcfc\guillemotright.

\newpage

\section*{Задача \rom{2}}

\begin{spacing}{1.5}
По номерам $\quad \text{\textnumero}^{\stackrel{A}{=}} (\alpha) = 1332 - 34 \cdot N 
\cdot (53 - n) \quad $ и $\quad \text{\textnumero}^{\stackrel{A}{=}} (\beta) = 
996 + 14 \cdot 
N \cdot (53 - n) \quad$ слов $\quad \alpha, \beta \in A^{+} \quad$ в в натуральном словаре позитивного 
языка над алфавитом \underline{\underline{A}} = $\left<a,b,c,d,e,f\right>$ 
определить номер слова $\text{\textnumero}^{\stackrel{A}{=}} (\alpha \otimes \beta)$ 
и проверить справедливость формулы:\\ 
$\text{\textnumero}^{\stackrel{A}{=}} 
(\alpha \otimes \beta) = |A|^{|\beta|} \cdot \text{\textnumero}^{\stackrel{A}{=}} 
(\alpha) + \text{\textnumero}^{\stackrel{A}{=}} (\beta)$
\end{spacing}

\section*{Дано}

\begin{equation*}
    N = 9, \quad n = 52, \quad 
    \text{\textnumero}^{\stackrel{A}{=}} (\alpha) = 1026, \quad
    \text{\textnumero}^{\stackrel{A}{=}} (\beta) = 1122
\end{equation*}

\section*{Решение}

Найдем оба слова, воспользовавшись алгоритмом из первой задачи

\newpage

\begin{align*}
    & \text{Алфавит}: [a, b, c, d, e, f] \\ 
    & \text{Длина алфавита}: 6
\end{align*}

\setlength{\columnsep}{20pt} % Optional: Adjusts the space between columns
\setlength{\columnseprule}{0.4pt} % Adds a vertical line between the columns

\begin{multicols}{2}
    % Альфа
    \begin{align*}
        \begin{aligned}
            & \text{Номер итерации: }&1\\
            & \text{Текущее значение: }&1026\\
            & 1026 \quad \% \quad 6 =& 0 \\
            & \text{Новое текущее значение: }&1020\\
            & \text{Текущий индекс буквы в словаре: }&5\\
            & \text{Текущая буква: }& \text{f}\\
            & \text{Текущее слово: }& \text{f}\\
        \end{aligned}
    \end{align*}

    \begin{align*}
        \begin{aligned}
            & \text{Номер итерации: }&2\\
            & \text{Текущее значение: }&170\\
            & 170 \quad \% \quad 6 =& 2 \\
            & \text{Текущий индекс буквы в словаре: }&1\\
            & \text{Текущая буква: }& \text{b}\\
            & \text{Текущее слово: }& \text{bf}\\
        \end{aligned}
    \end{align*}
    
    \begin{align*}
        \begin{aligned}
            & \text{Номер итерации: }&3\\
            & \text{Текущее значение: }&28\\
            & 28 \quad \% \quad 6 =& 4 \\
            & \text{Текущий индекс буквы в словаре: }&3\\
            & \text{Текущая буква: }& \text{d}\\
            & \text{Текущее слово: }& \text{dbf}\\
        \end{aligned}
    \end{align*}

    \begin{align*}
        \begin{aligned}
            & \text{Номер итерации: }&4\\
            & \text{Текущее значение: }&4\\
            & 4 \quad \% \quad 6 =& 4 \\
            & \text{Текущий индекс буквы в словаре: }&3\\
            & \text{Текущая буква: }& \text{d}\\
            & \text{Текущее слово: }& \text{ddbf}\\
        \end{aligned}
    \end{align*}\\
    
    \begin{center}
        Результат \guillemotleft ddbf\guillemotright\\
    \end{center}

    % Бета
    \columnbreak

    \begin{align*}
        \begin{aligned}
            & \text{Номер итерации: }&1\\
            & \text{Текущее значение: }&1122\\
            & 1122 \quad \% \quad 6 =& 0 \\
            & \text{Новое текущее значение: }&1116\\
            & \text{Текущий индекс буквы в словаре: }&5\\
            & \text{Текущая буква: }& \text{f}\\
            & \text{Текущее слово: }& \text{f}\\
        \end{aligned}
    \end{align*}

    \begin{align*}
        \begin{aligned}
            & \text{Номер итерации: }&2\\
            & \text{Текущее значение: }&186\\
            & 186 \quad \% \quad 6 =& 0 \\
            & \text{Новое текущее значение: }&180\\
            & \text{Текущий индекс буквы в словаре: }&5\\
            & \text{Текущая буква: }& \text{f}\\
            & \text{Текущее слово: }& \text{ff}\\
        \end{aligned}
    \end{align*}
    
    \begin{align*}
        \begin{aligned}
            & \text{Номер итерации: }&3\\
            & \text{Текущее значение: }&30\\
            & 30 \quad \% \quad 6 =& 0 \\
            & \text{Новое текущее значение: }&24\\
            & \text{Текущий индекс буквы в словаре: }&5\\
            & \text{Текущая буква: }& \text{f}\\
            & \text{Текущее слово: }& \text{fff}\\
        \end{aligned}
    \end{align*}

    \begin{align*}
        \begin{aligned}
            & \text{Номер итерации: }&4\\
            & \text{Текущее значение: }&4\\
            & 4 \quad \% \quad 6 =& 4 \\
            & \text{Текущий индекс буквы в словаре: }&3\\
            & \text{Текущая буква: }& \text{d}\\
            & \text{Текущее слово: }& \text{dfff}\\
        \end{aligned}
    \end{align*}\\

    \begin{center}
        Результат \guillemotleft dfff\guillemotright\\
    \end{center}
\end{multicols}

\noindent $\alpha \otimes \beta =$ ddbfdfff \\

\noindentСнова воспользовавшись \high{itertools} найдем:\\

\noindent $\text{\textnumero}^{\stackrel{A}{=}} (\alpha \otimes \beta)$ = 
1330818\\

\noindent \textbf{Проверка:}\\

\noindent $\text{\textnumero}^{\stackrel{A}{=}} (\alpha \otimes \beta) = 
6^4 \cdot 1026 + 1122 = 1330818$\\\\\\

\section*{Задача \rom{3}}

\begin{spacing}{1.5}
\noindent Минимизировать автомат по состояниям. Если исходный автомат не окажется 
приведённым, то изобразить диаграмму минимизированного автомата 
и произвести частичную проверку корректности минимизации, используя входную 
строку не менее чем из 5 (пяти) знаков входного алфавита и соответствующие 
начальные состояния исходного и минимизированного автоматов. 
Кроме того, доказать приведённость минимизированного автомата.
\end{spacing}

\section*{Дано}

\renewcommand{\arraystretch}{1.5} % Increase row height
\begin{table}[h!]
    \centering
    \begin{tabular}{|c|
        >{\centering\arraybackslash}p{1cm}|
        >{\centering\arraybackslash}p{1cm}|
        >{\centering\arraybackslash}p{1cm}|
        >{\centering\arraybackslash}p{1cm}|
        >{\centering\arraybackslash}p{1cm}|
        >{\centering\arraybackslash}p{1cm}|}
        \hline
        & \multicolumn{2}{c|}{a} & \multicolumn{2}{c|}{b} & \multicolumn{2}{c|}{c} \\
        \hline
        1 & 8 & b & 2 & c & 4 & c \\
        \hline
        2 & 5 & a & 1 & b & 8 & a \\
        \hline
        3 & 4 & c & 8 & a & 3 & a \\
        \hline
        4 & 7 & c & 1 & a & 5 & a \\
        \hline
        5 & 3 & c & 1 & a & 4 & a \\
        \hline
        6 & 7 & a & 8 & b & 1 & a \\
        \hline
        7 & 5 & c & 8 & a & 7 & a \\
        \hline
        8 & 1 & b & 6 & c & 3 & c \\
        \hline
    \end{tabular}
\end{table}

\newpage

Изобразим графически схему состояний автомата.

\begin{center}
    \begin{tikzpicture}[shorten >=1pt,node distance=3cm,on grid,auto,scale=1.2,transform shape]
        \node[state] (1) {1};
        \node[state] (2) [right=of 1] {2};
        \node[state] (3) [right=of 2] {3};
        \node[state] (4) [below=of 1] {4};
        \node[state] (5) [below=of 2] {5};
        \node[state] (6) [below=of 3] {6};
        \node[state] (7) [below=of 4] {7};
        \node[state] (8) [below=of 5] {8};
    
        \path[->]
        (1) edge [bend left=20]  node {a,b} (8)
            edge [bend left=20]  node {b,c} (2)
            edge [bend right=20] node [swap] {c,c} (4)
        (2) edge [bend left=20]  node {a,a} (5)
            edge [bend left=20]  node [swap] {b,b} (1)
            edge [bend right=20] node [swap] {c,a} (8)
        (3) edge [bend left=75]  node [swap] {a,c} (4)
            edge [bend left=20]  node {b,a} (8)
            edge [loop above]    node {c,a} ()
        (4) edge [bend right=20] node [swap] {a,c} (7)
            edge [bend left=20]  node [swap] {b,a} (1)
            edge [bend right=20] node [swap] {c,a} (5)
        (5) edge [bend left=20]  node {a,c} (3)
            edge [bend left=20]  node [swap] {b,a} (1)
            edge [bend right=20] node [swap] {c,a} (4)
        (6) edge [bend left=70]  node {a,a} (7)
            edge [bend left=40]  node [swap] {b,b} (8)
            edge [bend right=30] node [swap] {c,a} (1)
        (7) edge [bend left=20]  node [swap] {a,c} (5)
            edge [bend left=20]  node [swap] {b,a} (8)
            edge [loop below]    node {c,a} ()
        (8) edge [bend left=20]  node {a,b} (1)
            edge [bend left=35]  node [swap] {b,c} (6)
            edge [bend right=70] node [swap] {c,c} (3);
    \end{tikzpicture}
\end{center}

\section*{Решение}

\begin{spacing}{1.5}
\noindent \underline{\textbf{Первый шаг}}.
Отношение эквивалентности $\sim^e$ индуцирует разбиение $\boldsymbol{\pi} = 
\left<\pi_1, \pi_2, \pi_3\right>$
совокупность состояний $V = 
\left<1, 2, 3, 4, 5, 6, 7, 8\right>$ автомата Г на классы $\sim^e$ -эквивалентности
$\pi_1 = \left<1, 8\right>$, $\pi_2 = \left<2, 6\right>$, $\pi_3 = \left<3, 4, 5, 7\right>$, 
в каждом из которых буква выхода автомата определяется
входной буквой, не зависящей от состояния автомата из рассматриваемого класса
эквивалентности. Проиллюстрируем полученное разбиение $\boldsymbol{\pi} = 
\left<\pi_1, \pi_2, \pi_3\right>$.
\end{spacing}

\begin{gather*}
    \boldsymbol{\pi} = \left<\pi_1, \pi_2, \pi_3\right> \\
    \pi_1 = \left<1, 8\right> \quad 
    \pi_2 = \left<2, 6\right> \quad 
    \pi_3 = \left<3, 4, 5, 7\right>
\end{gather*}

\renewcommand{\arraystretch}{1.5} % Increase row height
\begin{table}[H]
    \centering
    \begin{tabular}{|c|
                    >{\centering\arraybackslash}p{1cm}|
                    >{\centering\arraybackslash}p{1cm}|
                    >{\centering\arraybackslash}p{1cm}|
                    >{\centering\arraybackslash}p{1cm}|
                    >{\centering\arraybackslash}p{1cm}|
                    >{\centering\arraybackslash}p{1cm}|
                     c}
        \cline{1-7}
        & \multicolumn{2}{c|}{a} & \multicolumn{2}{c|}{b} & \multicolumn{2}{c|}{c} & \\
        \cline{1-7}
        1 & 8 & b & 2 & c & 4 & c & \rdelim\}{2}{*}[$\pi_1$] \\
        \cline{1-7}
        8 & 1 & b & 6 & c & 3 & c & \\  
        \cline{1-7}
        2 & 5 & a & 1 & b & 8 & a & \rdelim\}{2}{*}[$\pi_2$] \\
        \cline{1-7}
        6 & 7 & a & 8 & b & 1 & a & \\  
        \cline{1-7}
        3 & 4 & c & 8 & a & 3 & a & \rdelim\}{4}{*}[$\pi_3$] \\
        \cline{1-7}
        4 & 7 & c & 1 & a & 5 & a & \\  
        \cline{1-7}
        5 & 3 & c & 1 & a & 4 & a & \\  
        \cline{1-7}
        7 & 5 & c & 8 & a & 7 & a & \\  
        \cline{1-7}
    \end{tabular}
\end{table}

\vspace{20pt}

\begin{spacing}{1.5}
\noindent \underline{\textbf{Второй шаг}}.
Отношение эквивалентности $\sim$ индуцирует разбиения $\boldsymbol{\pi}^{(1)} = (\Pi_1)$,
$\boldsymbol{\pi}^{(2)} = (\Pi_2)$ и $\boldsymbol{\pi}^{(3)} = (\Pi_3)$ на каждом из 
классов $\pi_1 = \left<1, 8\right>$, $\pi_2 = \left<2, 6\right>$, $\pi_3 = 
\left<3, 4, 5, 7\right>$, соответственно, где в каждом из классов $\Pi_1 = 
\left<1, 8\right>$, $\Pi_2 = \left<2, 6\right>$, $\Pi_3 = \left<3, 4, 5, 7\right>$ 
буква входа автомата переводит состояние автомата из этого класса в состояние одного 
из этих классов $\pi_1$, $\pi_2$ или $\pi_3$, зависящего только от этой входной буквы. 
Проиллюстрируем эти переходы.
\end{spacing}

\begin{equation*}
    \Pi_1 = \left<1, 8\right> \quad
    \Pi_2 = \left<2, 6\right> \quad
    \Pi_3 = \left<3, 4, 5, 7\right>
\end{equation*}

\renewcommand{\arraystretch}{1.5} % Increase row height
\begin{table}[H]
    \centering
    \begin{tabular}{c|
                    c|
                    >{\centering\arraybackslash}p{2cm}|
                    >{\centering\arraybackslash}p{2cm}|
                    >{\centering\arraybackslash}p{2cm}|
                    c}
        \cline{2-5}
        & & a & b & c & \\
        \cline{2-5}
        \ldelim\{{2}{*}[$\pi_1$] & 1 & $\pi_1$ & $\pi_2$ & $\pi_3$ & \rdelim\}{2}{*}[$\Pi_1$] \\
        \cline{2-5}
        & 8 & $\pi_1$ & $\pi_2$ & $\pi_3$ & \\
        \cline{2-5}
        \ldelim\{{2}{*}[$\pi_2$] & 2 & $\pi_3$ & $\pi_1$ & $\pi_1$ & \rdelim\}{2}{*}[$\Pi_2$] \\
        \cline{2-5}
        & 6 & $\pi_3$ & $\pi_1$ & $\pi_1$ & \\
        \cline{2-5}
        \ldelim\{{4}{*}[$\pi_3$] & 3 & $\pi_3$ & $\pi_1$ & $\pi_3$ & \rdelim\}{4}{*}[$\Pi_3$] \\
        \cline{2-5}
        & 4 & $\pi_3$ & $\pi_1$ & $\pi_3$ & \\
        \cline{2-5}
        & 5 & $\pi_3$ & $\pi_1$ & $\pi_3$ & \\
        \cline{2-5}
        & 7 & $\pi_3$ & $\pi_1$ & $\pi_3$ & \\
        \cline{2-5}
    \end{tabular}
\end{table}

\vspace{20pt}

\begin{spacing}{1.5}
    \noindent \underline{\textbf{Третий шаг}}.
     Так как $\boldsymbol{\pi} = \boldsymbol{\Pi}$ , то присваиваем
    разбиению $\boldsymbol{\pi}$ значение разбиения $\boldsymbol{\Pi}$ 
    ($\boldsymbol{\pi} := \boldsymbol{\Pi}$) и переходим к четвертому шагу алгоритма.
\end{spacing}

\begin{spacing}{1.5}
    \noindent \underline{\textbf{Четвертый шаг}}.
    Создаем автомат $\Omega = $Г/$\sim^a = (A, B, \boldsymbol{\pi},
    \overset{\sim}{\delta}, \overset{\sim}{\lambda}) = (A, B, \boldsymbol{\pi}, \ast, \bullet)$,
    где для состояния $v \in \pi_j (j = \overline{1, 4})$ из компоненты $\pi_j \in \boldsymbol{\pi}$:
    \begin{equation*}
        \begin{cases}
            \overset{\sim}{\delta}(a, \overset{\sim}{v}^a) = a \ast \overset{\sim}{v}^a =
            \delta(a, v); \\
            \overset{\sim}{\lambda}(a, \overset{\sim}{v}^a) = a \bullet \overset{\sim}{v}^a =
            \lambda(a, v).
        \end{cases}
    \end{equation*}
\end{spacing}

\begin{spacing}{1.5}
    \noindent \underline{\textbf{Пятый шаг}}.
    Выходим из алгоритма с результатом в виде приведенного автомата
    $\Omega = $Г/$\sim^a$, работу которого проиллюстрируем
\end{spacing}

\renewcommand{\arraystretch}{1.5} % Increase row height
\begin{table}[h!]
    \centering
    \begin{tabular}{|c|
        >{\centering\arraybackslash}p{1cm}|
        >{\centering\arraybackslash}p{1cm}|
        >{\centering\arraybackslash}p{1cm}|
        >{\centering\arraybackslash}p{1cm}|
        >{\centering\arraybackslash}p{1cm}|
        >{\centering\arraybackslash}p{1cm}|}
        \hline
        & \multicolumn{2}{c|}{a} & \multicolumn{2}{c|}{b} & \multicolumn{2}{c|}{c} \\
        \hline
        $\pi_1$ & $\pi_1$ & b & $\pi_2$ & c & $\pi_3$ & c \\
        \hline
        $\pi_2$ & $\pi_3$ & a & $\pi_1$ & b & $\pi_1$ & a \\
        \hline
        $\pi_3$ & $\pi_3$ & c & $\pi_1$ & a & $\pi_3$ & a \\
        \hline
    \end{tabular}
\end{table}

\vspace{10pt}

\noindent Изобразим графически схему состояний приведенного автомата.

\vspace{50pt}

\begin{center}
    \begin{tikzpicture}[shorten >=1pt,node distance=3cm,on grid,auto]
        \begin{scope}[scale=1.75, transform shape]
            \node[state] (pi_1)   {$\pi_1$};
            \node[state] (pi_2) [right=of pi_1] {$\pi_2$};
            \node[state] (pi_3) [below left=of pi_2] {$\pi_3$};
        
            \path[->]
            (pi_1) edge [loop above]   node {a,b} ()
                   edge [bend left=70] node {b,c} (pi_2)
                   edge [bend right=0] node [swap] {c,c} (pi_3)
            (pi_2) edge [bend left=20] node {a,a} (pi_3)
                   edge [bend left]    node [swap] {b,b} (pi_1)
                   edge [bend right]   node [swap] {c,a} (pi_1)
            (pi_3) edge [loop below]   node {a,c} ()
                   edge [bend left=60] node {b,a} (pi_1)
                   edge [loop right]   node {c,a} ();
        \end{scope}
    \end{tikzpicture}
\end{center}

\begin{center}
    \begin{tikzpicture}[shorten >=1pt,node distance=2cm,on grid,auto]
        \begin{scope}[scale=1.5, transform shape]
            \node[state] (pi_1_1)                   {$\pi_1$};
            \node[state] (pi_2)   [right=of pi_1_1] {$\pi_1$};
            \node[state] (pi_3)   [right=of pi_2]   {$\pi_2$};
            \node[state] (pi_1_2) [right=of pi_3]   {$\pi_1$};

            \path[->]
            % above arrows
            (pi_1_1) edge node {a} (pi_2)
            (pi_2)   edge node {b} (pi_3)
            (pi_3)   edge node {c} (pi_1_2);

            \path[->]
            % below arrows
            (pi_1_1) edge node [swap] {b} (pi_2)
            (pi_2)   edge node [swap] {c} (pi_3)
            (pi_3)   edge node [swap] {a} (pi_1_2);
        \end{scope}
    \end{tikzpicture}
\end{center}

\vspace{25pt}

\begin{center}
    \begin{tikzpicture}[shorten >=1pt,node distance=2cm,on grid,auto]
        \begin{scope}[scale=1.5, transform shape]
            \node[state] (pi_1_1)                   {$1$};
            \node[state] (pi_2)   [right=of pi_1_1] {$8$};
            \node[state] (pi_3)   [right=of pi_2]   {$6$};
            \node[state] (pi_1_2) [right=of pi_3]   {$1$};

            \path[->]
            % above arrows
            (pi_1_1) edge node {a} (pi_2)
            (pi_2)   edge node {b} (pi_3)
            (pi_3)   edge node {c} (pi_1_2);

            \path[->]
            % below arrows
            (pi_1_1) edge node [swap] {b} (pi_2)
            (pi_2)   edge node [swap] {c} (pi_3)
            (pi_3)   edge node [swap] {a} (pi_1_2);
        \end{scope}
    \end{tikzpicture}
\end{center}

\vspace{10pt}

\begin{center}
    abc $\cdot \pi_1$ = bca $\Longrightarrow$ abc $\cdot$ 1 = bca 
\end{center}

\vspace{10pt}

\noindent Доказательство приведенности автомата: $\Omega = 
(A, B, \boldsymbol{\pi}, \ast, \bullet)$

\vspace{10pt}

\begin{equation*}
    \begin{cases}
        a \cdot \pi_1 = b \\
        a \cdot \pi_2 = a \\
        a \cdot \pi_3 = c
    \end{cases} 
    \quad
    \Longrightarrow
    \quad
    \begin{cases}
        \pi_1 \cancel{\sim}^a \pi_2 \\
        \pi_1 \cancel{\sim}^a \pi_3 \\
        \pi_2 \cancel{\sim}^a \pi_3 
    \end{cases}
\end{equation*}

\end{document}
