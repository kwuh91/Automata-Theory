\documentclass[a4paper, 14pt]{extarticle}

\usepackage{../latexDependencies/misc/preamble2}

\geometry{a4paper}

% Название дисциплины
\newcommand{\subject}{Теория автоматов и алгоритмические языки} 

% Тип работы
% lab - для лабораторной работы 
% hw  - для домашней     работы
\newcommand{\task}{hw} 

% Номер работы
\newcommand{\taskNumber}{2-1} 

% % Название работы
% \newcommand{\taskNameOne}{Определение списка полиномов интерполяционного} 
% \newcommand{\taskNameTwo}{сплайна третьей степени единичного дефекта} 
% \newcommand{\taskNameThree}{для заданной сеточной функции} 

% Имя студента
\newcommand{\studentName}{Очкин Н.В.}

% Имя преподававателя
\newcommand{\teacherName}{Кутыркин В.А.}

% Группа
\newcommand{\group}{ФН11-52Б}

% Вариант
\newcommand{\variant}{9}

\begin{document}

\graphicspath{ {../latexDependencies/images} } 
\input{../latexDependencies/frontmatter/titlepage2}

\newgeometry{left=25mm, right=25mm, top=10mm, bottom=13mm}

\graphicspath{ {../latexDependencies/images/HW2-1} }

% Customize section, subsection, subsubsection and paragraph styles
\titleformat{\section}
  {\normalfont\large\bfseries}{\thesection}{1em}{}

\titleformat{\subsection}
  {\normalfont\normalsize\bfseries}{\thesubsection}{1em}{}

\titleformat{\subsubsection}
  {\normalfont\small\bfseries}{\thesubsubsection}{1em}{}

\titleformat{\paragraph}
  {\small\small\bfseries}{\theparagraph}{1em}{}

% \thispagestyle{empty}

% \null\newpage

% \setcounter{tocdepth}{5}
% \setcounter{secnumdepth}{5}

% \pagenumbering{roman}

% \tableofcontents
% \newpage

% \pagenumbering{arabic}
% \setcounter{page}{1}

\setstretch{1}
\linespread{1.1}

\setlength{\parindent}{0pt}

\fontsize{12pt}{16pt}\selectfont

% \definecolor{myblue}{HTML}{0A88C2}
% \definecolor{myred}{HTML}{FF1B1C}
% \definecolor{mygreen}{HTML}{386641}

% \lstdefinestyle{mystyle}{
%     basicstyle=\ttfamily\footnotesize,
%     keywordstyle=\color{myblue},
%     stringstyle=\color{myred},
%     commentstyle=\color{green!50!black},
%     showstringspaces=false,
%     frame=leftline, 
%     framesep=10pt, 
% }

% % Set the style for Python code
% \lstset{style=mystyle, extendedchars=\true}
% --------------------------------------START--------------------------------------

\section*{Задание}\vspace{-20pt}\rule{\linewidth}{0.1mm}

Для право-линейной грамматики создать автомат-анализатор. 
Продукции грамматики приведены ниже в таблице. Затем, инвертировав 
правые части продукций грамматики, получить лево-линейную грамматику 
и создать для неё автомат-анализатор. Сделать частичную проверку языка 
право- и лево-линейных грамматик, используя для этого автомат грамматики и 
автомат-анализатор языка автомата грамматики. Написать соответствующие 
правила вывода слов языка.

\section*{Исходные данные}\vspace{-20pt}\rule{\linewidth}{0.1mm}

\begin{center}
  \fontsize{17.28pt}{20.736pt}\selectfont 
  \bfseries
  Задача для право-линейной грамматики
  % $ \text{Таблица} \hspace{15pt} \to \hspace{15pt} \text{продукций} \hspace{15pt} \to \hspace{15pt} \text{грамматик} $
\end{center}

\vspace{-5pt}

\begin{figure}[h]
  \centering
  \includegraphics[page=1, scale=6]{graphics/data2}
\end{figure}

% \newpage

\vspace{-5pt}

\begin{center}
  \fontsize{17.28pt}{20.736pt}\selectfont 
  \bfseries
  Построим автомат право-линейной грамматики
\end{center}

% \vspace{-20pt}

\begin{figure}[h]
  \centering
  \includegraphics[page=1, scale=3.65]{graphics/diagramm1}
\end{figure}

\newpage
\newgeometry{left=25mm, right=25mm, top=20mm, bottom=20mm}

\begin{center}
  Произведем редукцию автомата относительно бесплодного состояния
\end{center}

\begin{figure}[h]
  \centering
  \includegraphics[page=1, scale=3.65]{graphics/diagramm2}
\end{figure}

Произведём детерминацию (синим цветом отмечены заключительные состояния)



\end{document}

% \begin{center}
%     \fontsize{12pt}{14.4pt}\selectfont 
%     \bfseries
%     \begin{tikzpicture}
%         % Draw the square
%         \draw (0,0) rectangle (5,5);
    
%         % Write the text inside the square
%         \node[align=left] at (2.5, 4) {S --> aC | aA\hphantom{ | cE}};
%         \node[align=left] at (2.5, 3) {A --> bA | bS | a\hphantom{C}};
%         \node[align=left] at (2.5, 2) {B --> aB | aC | cE};
%         \node[align=left] at (2.5, 1) {C --> bA | bB | b\hphantom{S}};

%         % Draw a bigger square
%         \draw (-0.1,-0.1) rectangle (5.1,5.1);
%     \end{tikzpicture}
% \end{center}