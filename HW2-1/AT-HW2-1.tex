\documentclass[a4paper, 14pt]{extarticle}

\usepackage{../latexDependencies/misc/preamble2}

\geometry{a4paper}

% Название дисциплины
\newcommand{\subject}{Теория автоматов и алгоритмические языки} 

% Тип работы
% lab - для лабораторной работы 
% hw  - для домашней     работы
\newcommand{\task}{hw} 

% Номер работы
\newcommand{\taskNumber}{2-1} 

% % Название работы
% \newcommand{\taskNameOne}{Определение списка полиномов интерполяционного} 
% \newcommand{\taskNameTwo}{сплайна третьей степени единичного дефекта} 
% \newcommand{\taskNameThree}{для заданной сеточной функции} 

% Имя студента
\newcommand{\studentName}{Очкин Н.В.}

% Имя преподававателя
\newcommand{\teacherName}{Кутыркин В.А.}

% Группа
\newcommand{\group}{ФН11-52Б}

% Вариант
\newcommand{\variant}{9}

\let\emptyset\varnothing

\begin{document}

\graphicspath{ {../latexDependencies/images} } 
\input{../latexDependencies/frontmatter/titlepage2}

\newgeometry{left=25mm, right=25mm, top=20mm, bottom=20mm}

\graphicspath{ {../latexDependencies/images/HW2-1} }

% Customize section, subsection, subsubsection and paragraph styles
\titleformat{\section}
  {\normalfont\large\bfseries}{\thesection}{2em}{}

\titleformat{\subsection}
  {\normalfont\normalsize\bfseries}{\thesubsection}{2em}{}

\titleformat{\subsubsection}
  {\normalfont\small\bfseries}{\thesubsubsection}{2em}{}

\titleformat{\paragraph}
  {\small\small\bfseries}{\theparagraph}{2em}{}

% \thispagestyle{empty}

% \null\newpage

% \setcounter{tocdepth}{5}
% \setcounter{secnumdepth}{5}

% \pagenumbering{roman}

% \tableofcontents
% \newpage

% \pagenumbering{arabic}
% \setcounter{page}{1}

\setstretch{1}
\linespread{1.1}

\setlength{\parindent}{0pt}

\fontsize{12pt}{16pt}\selectfont

% \definecolor{myblue}{HTML}{0A88C2}
% \definecolor{myred}{HTML}{FF1B1C}
% \definecolor{mygreen}{HTML}{386641}

% \lstdefinestyle{mystyle}{
%     basicstyle=\ttfamily\footnotesize,
%     keywordstyle=\color{myblue},
%     stringstyle=\color{myred},
%     commentstyle=\color{green!50!black},
%     showstringspaces=false,
%     frame=leftline, 
%     framesep=10pt, 
% }

% % Set the style for Python code
% \lstset{style=mystyle, extendedchars=\true}
% --------------------------------------START--------------------------------------

\section*{Задание}\vspace{-20pt}\rule{\linewidth}{0.1mm}

Для право-линейной грамматики создать автомат-анализатор. 
Продукции грамматики приведены ниже в таблице. Затем, инвертировав 
правые части продукций грамматики, получить лево-линейную грамматику 
и создать для неё автомат-анализатор. Сделать частичную проверку языка 
право- и лево-линейных грамматик, используя для этого автомат грамматики и 
автомат-анализатор языка автомата грамматики. Написать соответствующие 
правила вывода слов языка.

\section*{Исходные данные}\vspace{-20pt}\rule{\linewidth}{0.1mm}

\begin{center}
  % \fontsize{17.28pt}{20.736pt}\selectfont 
  \normalsize
  \bfseries
  Задача для право-линейной грамматики
  % $ \text{Таблица} \hspace{15pt} \to \hspace{15pt} \text{продукций} \hspace{15pt} \to \hspace{15pt} \text{грамматик} $
\end{center}

\vspace{-5pt}

\begin{figure}[h]
  \centering
  \includegraphics[width=\textwidth,height=\textheight,keepaspectratio]{graphics/data2}
\end{figure}

\newpage

\vfill

\begin{center}
  \normalsize
  \bfseries
  Построим диаграмму автомата право-линейной грамматики
\end{center}

\begin{figure}[h]
  \centering
  \includegraphics[width=0.5925\textwidth,height=0.5925\textheight,keepaspectratio]{graphics/diagramm1}
\end{figure}

\begin{center}
  \normalsize
  \bfseries
  Произведем редукцию автомата относительно бесплодного состояния
\end{center}

\begin{figure}[h]
  \centering
  \includegraphics[width=0.5925\textwidth,height=0.5925\textheight,keepaspectratio]{graphics/diagramm2}
\end{figure}

\vfill

\newpage

\begin{center}
  \normalsize
  \bfseries
  Произведём детерминацию
\end{center}

\begin{center}
  \resizebox{\textwidth}{!}{$\displaystyle
    \begin{array}{c@{\hspace{4em}}c}
      \begin{aligned}
        q_0 & = \left\{ S \right\} : \\
        & \quad
        \begin{cases}
          \xrightarrow{\hphantom{abc}a\hphantom{abc}} \left\{ A \hspace{3pt} C \right\} & = q_1 \\
          \xrightarrow{\hphantom{abc}b\hphantom{abc}} \emptyset
        \end{cases} \\[2em]
        q_1 & = \left\{ A \hspace{3pt} C \right\} : \\
        &
        \begin{cases}
          \xrightarrow{\hphantom{abc}a\hphantom{abc}} \left\{ K \right\} & = q_2 \rightsquigarrow \\
          \xrightarrow{\hphantom{abc}b\hphantom{abc}} \left\{ A \hspace{3pt} S \hspace{3pt} K \right\} & = q_3 \rightsquigarrow
        \end{cases} \\[2em]
        q_2 & = \left\{ K \right\} \\
        & \xrightarrow{\hphantom{\hspace{107.5pt}}} q_2 \\[2em]
        q_3 & = \left\{ A \hspace{3pt} S \hspace{3pt} K \right\} : \\
        &
        \begin{cases}
          \xrightarrow{\hphantom{abc}a\hphantom{abc}} \left\{ K \hspace{3pt} C \hspace{3pt} A \right\} & = q_4 \rightsquigarrow \\
          \xrightarrow{\hphantom{abc}b\hphantom{abc}} \left\{ A \hspace{3pt} S \hspace{3pt} \right\} & = q_5
        \end{cases} \\[2em]
        q_4 & = \left\{ K \hspace{3pt} C \hspace{3pt} A \right\} : \\
        &
        \begin{cases}
          \xrightarrow{\hphantom{abc}a\hphantom{abc}} \left\{ K \right\} & = q_2 \\
          \xrightarrow{\hphantom{abc}b\hphantom{abc}} \left\{ A \hspace{3pt} K \hspace{3pt} B \hspace{3pt} S \right\} & = q_6 \rightsquigarrow
        \end{cases} \\[2em]
        q_5 & = \left\{ A \hspace{3pt} S \right\} : \\
        &
        \begin{cases}
          \xrightarrow{\hphantom{abc}a\hphantom{abc}} \left\{ K \hspace{3pt} C \hspace{3pt} A \right\} & = q_4 \\
          \xrightarrow{\hphantom{abc}b\hphantom{abc}} \left\{ A \hspace{3pt} S \right\} & = q_5
        \end{cases}
      \end{aligned}
      &
      \begin{aligned}
        q_6 & = \left\{ A \hspace{3pt} K \hspace{3pt} B \hspace{3pt} S \right\} : \\
        &
        \begin{cases}
          \xrightarrow{\hphantom{abc}a\hphantom{abc}} \left\{ K \hspace{3pt} C \hspace{3pt} B \hspace{3pt} A \right\} & = q_7 \rightsquigarrow \\
          \xrightarrow{\hphantom{abc}b\hphantom{abc}} \left\{ A \hspace{3pt} S \right\} & = q_5
        \end{cases} \\[2em]
        q_7 & = \left\{ K \hspace{3pt} C \hspace{3pt} B \hspace{3pt} A \right\} : \\
        &
        \begin{cases}
          \xrightarrow{\hphantom{abc}a\hphantom{abc}} \left\{ C \hspace{3pt} B \hspace{3pt} K \right\} & = q_8 \rightsquigarrow \\
          \xrightarrow{\hphantom{abc}b\hphantom{abc}} \left\{ A \hspace{3pt} K \hspace{3pt} S \right\} & = q_3
        \end{cases} \\[2em]
        q_8 & = \left\{ C \hspace{3pt} B \hspace{3pt} K \right\} : \\
        &
        \begin{cases}
          \xrightarrow{\hphantom{abc}a\hphantom{abc}} \left\{ C \hspace{3pt} B \right\} & = q_9 \\
          \xrightarrow{\hphantom{abc}b\hphantom{abc}} \left\{ K \right\} & = q_2
        \end{cases} \\[2em]
        q_9 & = \left\{ C \hspace{3pt} B \right\} : \\
        &
        \begin{cases}
          \xrightarrow{\hphantom{abc}a\hphantom{abc}} \left\{ C \hspace{3pt} B \right\} & = q_9 \\
          \xrightarrow{\hphantom{abc}b\hphantom{abc}} \left\{ A \hspace{3pt} K \hspace{3pt} B \right\} & = q_{10} \rightsquigarrow
        \end{cases} \\[2em]
        q_{10} & = \left\{ A \hspace{3pt} K \hspace{3pt} B \right\} : \\
        &
        \begin{cases}
          \xrightarrow{\hphantom{abc}a\hphantom{abc}} \left\{ K \hspace{3pt} C \hspace{3pt} B \right\} & = q_8 \\
          \xrightarrow{\hphantom{abc}b\hphantom{abc}} \left\{ A \hspace{3pt} S \right\} & = q_5
        \end{cases}
      \end{aligned}
    \end{array}$
  }
\end{center}

\newpage

\begin{center}
  \normalsize
  \bfseries
  Автомат-анализатор для право-линейной грамматики
\end{center}

\begin{figure}[h]
  \centering
  \includegraphics[width=\textwidth,height=\textheight,keepaspectratio]{graphics/diagramm3}
\end{figure}

\vfill

\begin{center}
  \normalsize
  \bfseries
  Выполним проверку
\end{center}

\begin{figure}[h]
  \centering
  \includegraphics[width=0.5\textwidth,height=0.5\textheight,keepaspectratio]{graphics/diagramm4}
\end{figure}

\begin{center}
  \normalsize
  \bfseries
  Проверка сошлась для автомата грамматики и автомата-анализатора языка право-линейной грамматики.
\end{center}

\newpage

\vfill

\begin{center}
  \normalsize
  \bfseries
  Задача для лево-линейной грамматики
\end{center}

\vspace{-5pt}

\begin{figure}[h]
  \centering
  \includegraphics[width=0.7\textwidth,height=0.7\textheight,keepaspectratio]{one}
\end{figure}

\vfill

\begin{center}
  \normalsize
  \bfseries
  Построим диаграмму автомата лево-линейной грамматики
\end{center}

\begin{figure}[h]
  \centering
  \includegraphics[width=0.85\textwidth,height=0.85\textheight,keepaspectratio]{two}
\end{figure}

\vfill

\newpage

\begin{center}
  \normalsize
  \bfseries
  Произведем редукцию автомата относительно недостижимого состояния
\end{center}

\begin{figure}[h]
  \centering
  \includegraphics[width=1\textwidth,height=1\textheight,keepaspectratio]{three}
\end{figure}

\newpage

\begin{center}
  \normalsize
  \bfseries
  Произведём детерминацию
\end{center}

\begin{figure}[h]
  \centering
  \includegraphics[width=0.9\textwidth,height=0.9\textheight,keepaspectratio]{four}
\end{figure}

\newpage

\begin{center}
  \normalsize
  \bfseries
  Автомат-анализатор для лево-линейной грамматики
\end{center}

\begin{figure}[h]
  \centering
  \includegraphics[width=0.7\textwidth,height=0.7\textheight,keepaspectratio]{five}
\end{figure}

\begin{center}
  \normalsize
  \bfseries
  Выполним проверку
\end{center}

\begin{figure}[h]
  \centering
  \includegraphics[width=0.7\textwidth,height=0.7\textheight,keepaspectratio]{six}
\end{figure}

\begin{center}
  \normalsize
  \bfseries
  Проверка сошлась для автомата грамматики и автомата-анализатора языка лево-линейной грамматики.
\end{center}

\end{document}

% \begin{center}
%     \fontsize{12pt}{14.4pt}\selectfont 
%     \bfseries
%     \begin{tikzpicture}
%         % Draw the square
%         \draw (0,0) rectangle (5,5);
    
%         % Write the text inside the square
%         \node[align=left] at (2.5, 4) {S --> aC | aA\hphantom{ | cE}};
%         \node[align=left] at (2.5, 3) {A --> bA | bS | a\hphantom{C}};
%         \node[align=left] at (2.5, 2) {B --> aB | aC | cE};
%         \node[align=left] at (2.5, 1) {C --> bA | bB | b\hphantom{S}};

%         % Draw a bigger square
%         \draw (-0.1,-0.1) rectangle (5.1,5.1);
%     \end{tikzpicture}
% \end{center}